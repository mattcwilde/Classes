
% Default to the notebook output style

    


% Inherit from the specified cell style.




    
\documentclass{article}

    
    
    \usepackage{graphicx} % Used to insert images
    \usepackage{adjustbox} % Used to constrain images to a maximum size 
    \usepackage{color} % Allow colors to be defined
    \usepackage{enumerate} % Needed for markdown enumerations to work
    \usepackage{geometry} % Used to adjust the document margins
    \usepackage{amsmath} % Equations
    \usepackage{amssymb} % Equations
    \usepackage{eurosym} % defines \euro
    \usepackage[mathletters]{ucs} % Extended unicode (utf-8) support
    \usepackage[utf8x]{inputenc} % Allow utf-8 characters in the tex document
    \usepackage{fancyvrb} % verbatim replacement that allows latex
    \usepackage{grffile} % extends the file name processing of package graphics 
                         % to support a larger range 
    % The hyperref package gives us a pdf with properly built
    % internal navigation ('pdf bookmarks' for the table of contents,
    % internal cross-reference links, web links for URLs, etc.)
    \usepackage{hyperref}
    \usepackage{longtable} % longtable support required by pandoc >1.10
    \usepackage{booktabs}  % table support for pandoc > 1.12.2
    

    
    
    \definecolor{orange}{cmyk}{0,0.4,0.8,0.2}
    \definecolor{darkorange}{rgb}{.71,0.21,0.01}
    \definecolor{darkgreen}{rgb}{.12,.54,.11}
    \definecolor{myteal}{rgb}{.26, .44, .56}
    \definecolor{gray}{gray}{0.45}
    \definecolor{lightgray}{gray}{.95}
    \definecolor{mediumgray}{gray}{.8}
    \definecolor{inputbackground}{rgb}{.95, .95, .85}
    \definecolor{outputbackground}{rgb}{.95, .95, .95}
    \definecolor{traceback}{rgb}{1, .95, .95}
    % ansi colors
    \definecolor{red}{rgb}{.6,0,0}
    \definecolor{green}{rgb}{0,.65,0}
    \definecolor{brown}{rgb}{0.6,0.6,0}
    \definecolor{blue}{rgb}{0,.145,.698}
    \definecolor{purple}{rgb}{.698,.145,.698}
    \definecolor{cyan}{rgb}{0,.698,.698}
    \definecolor{lightgray}{gray}{0.5}
    
    % bright ansi colors
    \definecolor{darkgray}{gray}{0.25}
    \definecolor{lightred}{rgb}{1.0,0.39,0.28}
    \definecolor{lightgreen}{rgb}{0.48,0.99,0.0}
    \definecolor{lightblue}{rgb}{0.53,0.81,0.92}
    \definecolor{lightpurple}{rgb}{0.87,0.63,0.87}
    \definecolor{lightcyan}{rgb}{0.5,1.0,0.83}
    
    % commands and environments needed by pandoc snippets
    % extracted from the output of `pandoc -s`
    \providecommand{\tightlist}{%
      \setlength{\itemsep}{0pt}\setlength{\parskip}{0pt}}
    \DefineVerbatimEnvironment{Highlighting}{Verbatim}{commandchars=\\\{\}}
    % Add ',fontsize=\small' for more characters per line
    \newenvironment{Shaded}{}{}
    \newcommand{\KeywordTok}[1]{\textcolor[rgb]{0.00,0.44,0.13}{\textbf{{#1}}}}
    \newcommand{\DataTypeTok}[1]{\textcolor[rgb]{0.56,0.13,0.00}{{#1}}}
    \newcommand{\DecValTok}[1]{\textcolor[rgb]{0.25,0.63,0.44}{{#1}}}
    \newcommand{\BaseNTok}[1]{\textcolor[rgb]{0.25,0.63,0.44}{{#1}}}
    \newcommand{\FloatTok}[1]{\textcolor[rgb]{0.25,0.63,0.44}{{#1}}}
    \newcommand{\CharTok}[1]{\textcolor[rgb]{0.25,0.44,0.63}{{#1}}}
    \newcommand{\StringTok}[1]{\textcolor[rgb]{0.25,0.44,0.63}{{#1}}}
    \newcommand{\CommentTok}[1]{\textcolor[rgb]{0.38,0.63,0.69}{\textit{{#1}}}}
    \newcommand{\OtherTok}[1]{\textcolor[rgb]{0.00,0.44,0.13}{{#1}}}
    \newcommand{\AlertTok}[1]{\textcolor[rgb]{1.00,0.00,0.00}{\textbf{{#1}}}}
    \newcommand{\FunctionTok}[1]{\textcolor[rgb]{0.02,0.16,0.49}{{#1}}}
    \newcommand{\RegionMarkerTok}[1]{{#1}}
    \newcommand{\ErrorTok}[1]{\textcolor[rgb]{1.00,0.00,0.00}{\textbf{{#1}}}}
    \newcommand{\NormalTok}[1]{{#1}}
    
    % Define a nice break command that doesn't care if a line doesn't already
    % exist.
    \def\br{\hspace*{\fill} \\* }
    % Math Jax compatability definitions
    \def\gt{>}
    \def\lt{<}
    % Document parameters
    \title{Lab3}
    \author{Matt Wilde}
    
    
    

    % Pygments definitions
    
\makeatletter
\def\PY@reset{\let\PY@it=\relax \let\PY@bf=\relax%
    \let\PY@ul=\relax \let\PY@tc=\relax%
    \let\PY@bc=\relax \let\PY@ff=\relax}
\def\PY@tok#1{\csname PY@tok@#1\endcsname}
\def\PY@toks#1+{\ifx\relax#1\empty\else%
    \PY@tok{#1}\expandafter\PY@toks\fi}
\def\PY@do#1{\PY@bc{\PY@tc{\PY@ul{%
    \PY@it{\PY@bf{\PY@ff{#1}}}}}}}
\def\PY#1#2{\PY@reset\PY@toks#1+\relax+\PY@do{#2}}

\expandafter\def\csname PY@tok@gd\endcsname{\def\PY@tc##1{\textcolor[rgb]{0.63,0.00,0.00}{##1}}}
\expandafter\def\csname PY@tok@gu\endcsname{\let\PY@bf=\textbf\def\PY@tc##1{\textcolor[rgb]{0.50,0.00,0.50}{##1}}}
\expandafter\def\csname PY@tok@gt\endcsname{\def\PY@tc##1{\textcolor[rgb]{0.00,0.27,0.87}{##1}}}
\expandafter\def\csname PY@tok@gs\endcsname{\let\PY@bf=\textbf}
\expandafter\def\csname PY@tok@gr\endcsname{\def\PY@tc##1{\textcolor[rgb]{1.00,0.00,0.00}{##1}}}
\expandafter\def\csname PY@tok@cm\endcsname{\let\PY@it=\textit\def\PY@tc##1{\textcolor[rgb]{0.25,0.50,0.50}{##1}}}
\expandafter\def\csname PY@tok@vg\endcsname{\def\PY@tc##1{\textcolor[rgb]{0.10,0.09,0.49}{##1}}}
\expandafter\def\csname PY@tok@m\endcsname{\def\PY@tc##1{\textcolor[rgb]{0.40,0.40,0.40}{##1}}}
\expandafter\def\csname PY@tok@mh\endcsname{\def\PY@tc##1{\textcolor[rgb]{0.40,0.40,0.40}{##1}}}
\expandafter\def\csname PY@tok@go\endcsname{\def\PY@tc##1{\textcolor[rgb]{0.53,0.53,0.53}{##1}}}
\expandafter\def\csname PY@tok@ge\endcsname{\let\PY@it=\textit}
\expandafter\def\csname PY@tok@vc\endcsname{\def\PY@tc##1{\textcolor[rgb]{0.10,0.09,0.49}{##1}}}
\expandafter\def\csname PY@tok@il\endcsname{\def\PY@tc##1{\textcolor[rgb]{0.40,0.40,0.40}{##1}}}
\expandafter\def\csname PY@tok@cs\endcsname{\let\PY@it=\textit\def\PY@tc##1{\textcolor[rgb]{0.25,0.50,0.50}{##1}}}
\expandafter\def\csname PY@tok@cp\endcsname{\def\PY@tc##1{\textcolor[rgb]{0.74,0.48,0.00}{##1}}}
\expandafter\def\csname PY@tok@gi\endcsname{\def\PY@tc##1{\textcolor[rgb]{0.00,0.63,0.00}{##1}}}
\expandafter\def\csname PY@tok@gh\endcsname{\let\PY@bf=\textbf\def\PY@tc##1{\textcolor[rgb]{0.00,0.00,0.50}{##1}}}
\expandafter\def\csname PY@tok@ni\endcsname{\let\PY@bf=\textbf\def\PY@tc##1{\textcolor[rgb]{0.60,0.60,0.60}{##1}}}
\expandafter\def\csname PY@tok@nl\endcsname{\def\PY@tc##1{\textcolor[rgb]{0.63,0.63,0.00}{##1}}}
\expandafter\def\csname PY@tok@nn\endcsname{\let\PY@bf=\textbf\def\PY@tc##1{\textcolor[rgb]{0.00,0.00,1.00}{##1}}}
\expandafter\def\csname PY@tok@no\endcsname{\def\PY@tc##1{\textcolor[rgb]{0.53,0.00,0.00}{##1}}}
\expandafter\def\csname PY@tok@na\endcsname{\def\PY@tc##1{\textcolor[rgb]{0.49,0.56,0.16}{##1}}}
\expandafter\def\csname PY@tok@nb\endcsname{\def\PY@tc##1{\textcolor[rgb]{0.00,0.50,0.00}{##1}}}
\expandafter\def\csname PY@tok@nc\endcsname{\let\PY@bf=\textbf\def\PY@tc##1{\textcolor[rgb]{0.00,0.00,1.00}{##1}}}
\expandafter\def\csname PY@tok@nd\endcsname{\def\PY@tc##1{\textcolor[rgb]{0.67,0.13,1.00}{##1}}}
\expandafter\def\csname PY@tok@ne\endcsname{\let\PY@bf=\textbf\def\PY@tc##1{\textcolor[rgb]{0.82,0.25,0.23}{##1}}}
\expandafter\def\csname PY@tok@nf\endcsname{\def\PY@tc##1{\textcolor[rgb]{0.00,0.00,1.00}{##1}}}
\expandafter\def\csname PY@tok@si\endcsname{\let\PY@bf=\textbf\def\PY@tc##1{\textcolor[rgb]{0.73,0.40,0.53}{##1}}}
\expandafter\def\csname PY@tok@s2\endcsname{\def\PY@tc##1{\textcolor[rgb]{0.73,0.13,0.13}{##1}}}
\expandafter\def\csname PY@tok@vi\endcsname{\def\PY@tc##1{\textcolor[rgb]{0.10,0.09,0.49}{##1}}}
\expandafter\def\csname PY@tok@nt\endcsname{\let\PY@bf=\textbf\def\PY@tc##1{\textcolor[rgb]{0.00,0.50,0.00}{##1}}}
\expandafter\def\csname PY@tok@nv\endcsname{\def\PY@tc##1{\textcolor[rgb]{0.10,0.09,0.49}{##1}}}
\expandafter\def\csname PY@tok@s1\endcsname{\def\PY@tc##1{\textcolor[rgb]{0.73,0.13,0.13}{##1}}}
\expandafter\def\csname PY@tok@kd\endcsname{\let\PY@bf=\textbf\def\PY@tc##1{\textcolor[rgb]{0.00,0.50,0.00}{##1}}}
\expandafter\def\csname PY@tok@sh\endcsname{\def\PY@tc##1{\textcolor[rgb]{0.73,0.13,0.13}{##1}}}
\expandafter\def\csname PY@tok@sc\endcsname{\def\PY@tc##1{\textcolor[rgb]{0.73,0.13,0.13}{##1}}}
\expandafter\def\csname PY@tok@sx\endcsname{\def\PY@tc##1{\textcolor[rgb]{0.00,0.50,0.00}{##1}}}
\expandafter\def\csname PY@tok@bp\endcsname{\def\PY@tc##1{\textcolor[rgb]{0.00,0.50,0.00}{##1}}}
\expandafter\def\csname PY@tok@c1\endcsname{\let\PY@it=\textit\def\PY@tc##1{\textcolor[rgb]{0.25,0.50,0.50}{##1}}}
\expandafter\def\csname PY@tok@kc\endcsname{\let\PY@bf=\textbf\def\PY@tc##1{\textcolor[rgb]{0.00,0.50,0.00}{##1}}}
\expandafter\def\csname PY@tok@c\endcsname{\let\PY@it=\textit\def\PY@tc##1{\textcolor[rgb]{0.25,0.50,0.50}{##1}}}
\expandafter\def\csname PY@tok@mf\endcsname{\def\PY@tc##1{\textcolor[rgb]{0.40,0.40,0.40}{##1}}}
\expandafter\def\csname PY@tok@err\endcsname{\def\PY@bc##1{\setlength{\fboxsep}{0pt}\fcolorbox[rgb]{1.00,0.00,0.00}{1,1,1}{\strut ##1}}}
\expandafter\def\csname PY@tok@mb\endcsname{\def\PY@tc##1{\textcolor[rgb]{0.40,0.40,0.40}{##1}}}
\expandafter\def\csname PY@tok@ss\endcsname{\def\PY@tc##1{\textcolor[rgb]{0.10,0.09,0.49}{##1}}}
\expandafter\def\csname PY@tok@sr\endcsname{\def\PY@tc##1{\textcolor[rgb]{0.73,0.40,0.53}{##1}}}
\expandafter\def\csname PY@tok@mo\endcsname{\def\PY@tc##1{\textcolor[rgb]{0.40,0.40,0.40}{##1}}}
\expandafter\def\csname PY@tok@kn\endcsname{\let\PY@bf=\textbf\def\PY@tc##1{\textcolor[rgb]{0.00,0.50,0.00}{##1}}}
\expandafter\def\csname PY@tok@mi\endcsname{\def\PY@tc##1{\textcolor[rgb]{0.40,0.40,0.40}{##1}}}
\expandafter\def\csname PY@tok@gp\endcsname{\let\PY@bf=\textbf\def\PY@tc##1{\textcolor[rgb]{0.00,0.00,0.50}{##1}}}
\expandafter\def\csname PY@tok@o\endcsname{\def\PY@tc##1{\textcolor[rgb]{0.40,0.40,0.40}{##1}}}
\expandafter\def\csname PY@tok@kr\endcsname{\let\PY@bf=\textbf\def\PY@tc##1{\textcolor[rgb]{0.00,0.50,0.00}{##1}}}
\expandafter\def\csname PY@tok@s\endcsname{\def\PY@tc##1{\textcolor[rgb]{0.73,0.13,0.13}{##1}}}
\expandafter\def\csname PY@tok@kp\endcsname{\def\PY@tc##1{\textcolor[rgb]{0.00,0.50,0.00}{##1}}}
\expandafter\def\csname PY@tok@w\endcsname{\def\PY@tc##1{\textcolor[rgb]{0.73,0.73,0.73}{##1}}}
\expandafter\def\csname PY@tok@kt\endcsname{\def\PY@tc##1{\textcolor[rgb]{0.69,0.00,0.25}{##1}}}
\expandafter\def\csname PY@tok@ow\endcsname{\let\PY@bf=\textbf\def\PY@tc##1{\textcolor[rgb]{0.67,0.13,1.00}{##1}}}
\expandafter\def\csname PY@tok@sb\endcsname{\def\PY@tc##1{\textcolor[rgb]{0.73,0.13,0.13}{##1}}}
\expandafter\def\csname PY@tok@k\endcsname{\let\PY@bf=\textbf\def\PY@tc##1{\textcolor[rgb]{0.00,0.50,0.00}{##1}}}
\expandafter\def\csname PY@tok@se\endcsname{\let\PY@bf=\textbf\def\PY@tc##1{\textcolor[rgb]{0.73,0.40,0.13}{##1}}}
\expandafter\def\csname PY@tok@sd\endcsname{\let\PY@it=\textit\def\PY@tc##1{\textcolor[rgb]{0.73,0.13,0.13}{##1}}}

\def\PYZbs{\char`\\}
\def\PYZus{\char`\_}
\def\PYZob{\char`\{}
\def\PYZcb{\char`\}}
\def\PYZca{\char`\^}
\def\PYZam{\char`\&}
\def\PYZlt{\char`\<}
\def\PYZgt{\char`\>}
\def\PYZsh{\char`\#}
\def\PYZpc{\char`\%}
\def\PYZdl{\char`\$}
\def\PYZhy{\char`\-}
\def\PYZsq{\char`\'}
\def\PYZdq{\char`\"}
\def\PYZti{\char`\~}
% for compatibility with earlier versions
\def\PYZat{@}
\def\PYZlb{[}
\def\PYZrb{]}
\makeatother


    % Exact colors from NB
    \definecolor{incolor}{rgb}{0.0, 0.0, 0.5}
    \definecolor{outcolor}{rgb}{0.545, 0.0, 0.0}



    
    % Prevent overflowing lines due to hard-to-break entities
    \sloppy 
    % Setup hyperref package
    \hypersetup{
      breaklinks=true,  % so long urls are correctly broken across lines
      colorlinks=true,
      urlcolor=blue,
      linkcolor=darkorange,
      citecolor=darkgreen,
      }
    % Slightly bigger margins than the latex defaults
    
    \geometry{verbose,tmargin=1in,bmargin=1in,lmargin=1in,rmargin=1in}
    
    

    \begin{document}
    
    
    \maketitle
    
    \section*{3  Fraunhofer line strengths and the curve of growth}

    
    \begin{Verbatim}[commandchars=\\\{\}]
{\color{incolor}In [{\color{incolor}1}]:} \PY{o}{\PYZpc{}}\PY{k}{matplotlib} inline
        \PY{k+kn}{import} \PY{n+nn}{numpy} \PY{k+kn}{as} \PY{n+nn}{np}
        \PY{k+kn}{import} \PY{n+nn}{matplotlib.pyplot} \PY{k+kn}{as} \PY{n+nn}{plt}
        \PY{k+kn}{import} \PY{n+nn}{matplotlib} \PY{k+kn}{as} \PY{n+nn}{mpl}
        \PY{k+kn}{from} \PY{n+nn}{\PYZus{}\PYZus{}future\PYZus{}\PYZus{}} \PY{k+kn}{import} \PY{n}{division}
        \PY{k+kn}{import} \PY{n+nn}{astropy.constants} \PY{k+kn}{as} \PY{n+nn}{const}
        
        \PY{c}{\PYZsh{}mpl.rcParams.update(\PYZob{}\PYZsq{}text.usetex\PYZsq{}: True\PYZcb{})}
\end{Verbatim}

    \begin{Verbatim}[commandchars=\\\{\}]
{\color{incolor}In [{\color{incolor}2}]:} \PY{n}{h} \PY{o}{=} \PY{n}{const}\PY{o}{.}\PY{n}{h}\PY{o}{.}\PY{n}{cgs}\PY{o}{.}\PY{n}{value}
        \PY{n}{c} \PY{o}{=} \PY{n}{const}\PY{o}{.}\PY{n}{c}\PY{o}{.}\PY{n}{cgs}\PY{o}{.}\PY{n}{value}
        \PY{n}{k} \PY{o}{=} \PY{n}{const}\PY{o}{.}\PY{n}{k\PYZus{}B}\PY{o}{.}\PY{n}{cgs}\PY{o}{.}\PY{n}{value}
\end{Verbatim}

    \textbf{Write an IDL function planck,temp,wav in cgs units. The required
constants are given in Table 2 on page 14. For temp=5000 and wav=5000e-8
(5000 Angstrom), in the yellow part of the visible wavelength region and
at about the sensitivity peak of your eyes) it should give:}

    \begin{Verbatim}[commandchars=\\\{\}]
{\color{incolor}In [{\color{incolor}3}]:} \PY{k}{def} \PY{n+nf}{planck}\PY{p}{(}\PY{n}{temp}\PY{p}{,} \PY{n}{wav}\PY{p}{)}\PY{p}{:}
            \PY{n}{b} \PY{o}{=} \PY{p}{(}\PY{l+m+mi}{2}\PY{o}{*}\PY{n}{h}\PY{o}{*}\PY{n}{c}\PY{o}{*}\PY{o}{*}\PY{l+m+mi}{2}\PY{o}{/}\PY{n}{wav}\PY{o}{*}\PY{o}{*}\PY{l+m+mi}{5}\PY{p}{)} \PY{o}{/} \PY{p}{(}\PY{n}{np}\PY{o}{.}\PY{n}{exp}\PY{p}{(}\PY{p}{(}\PY{n}{h}\PY{o}{*}\PY{n}{c}\PY{p}{)}\PY{o}{/}\PY{p}{(}\PY{n}{wav}\PY{o}{*}\PY{n}{k}\PY{o}{*}\PY{n}{temp}\PY{p}{)}\PY{p}{)} \PY{o}{\PYZhy{}} \PY{l+m+mi}{1}\PY{p}{)}
            \PY{k}{return} \PY{n}{b}
\end{Verbatim}

    \begin{Verbatim}[commandchars=\\\{\}]
{\color{incolor}In [{\color{incolor}4}]:} \PY{k}{print} \PY{n}{planck}\PY{p}{(}\PY{l+m+mi}{5000}\PY{p}{,}\PY{l+m+mf}{5000e\PYZhy{}8}\PY{p}{)}
\end{Verbatim}

    \begin{Verbatim}[commandchars=\\\{\}]
1.21071855235e+14
    \end{Verbatim}

    \textbf{Use it to plot Planck curves against wavelength in the visble
part of the spectrum for different stellarlike temperatures, for example
with the following statements in a main IDL file SSA3.PRO:}

    \begin{Verbatim}[commandchars=\\\{\}]
{\color{incolor}In [{\color{incolor}5}]:} \PY{n}{wav} \PY{o}{=} \PY{n}{np}\PY{o}{.}\PY{n}{arange}\PY{p}{(}\PY{l+m+mi}{100}\PY{p}{)}\PY{o}{*}\PY{l+m+mf}{200.} \PY{o}{+} \PY{l+m+mf}{1000.}
\end{Verbatim}

    \begin{Verbatim}[commandchars=\\\{\}]
{\color{incolor}In [{\color{incolor}6}]:} \PY{n}{b} \PY{o}{=} \PY{n}{np}\PY{o}{.}\PY{n}{zeros\PYZus{}like}\PY{p}{(}\PY{n}{wav}\PY{p}{)}
        
        \PY{k}{for} \PY{n}{i} \PY{o+ow}{in} \PY{n+nb}{range}\PY{p}{(}\PY{l+m+mi}{100}\PY{p}{)}\PY{p}{:}
            \PY{n}{b}\PY{p}{[}\PY{n}{i}\PY{p}{]} \PY{o}{=} \PY{n}{planck}\PY{p}{(}\PY{l+m+mi}{8000}\PY{p}{,}\PY{n}{wav}\PY{p}{[}\PY{n}{i}\PY{p}{]}\PY{o}{*}\PY{l+m+mf}{1e\PYZhy{}8}\PY{p}{)}
        
        \PY{n}{plt}\PY{o}{.}\PY{n}{plot}\PY{p}{(}\PY{n}{wav}\PY{p}{,}\PY{n}{b}\PY{p}{)}
        \PY{n}{plt}\PY{o}{.}\PY{n}{xlabel}\PY{p}{(}\PY{l+s}{r\PYZsq{}}\PY{l+s}{wavelength (Angstroms)}\PY{l+s}{\PYZsq{}}\PY{p}{)}
        \PY{n}{plt}\PY{o}{.}\PY{n}{ylabel}\PY{p}{(}\PY{l+s}{r\PYZsq{}}\PY{l+s}{Placnk function}\PY{l+s}{\PYZsq{}}\PY{p}{)}
        \PY{n}{plt}\PY{o}{.}\PY{n}{xlim}\PY{p}{(}\PY{p}{[}\PY{n}{wav}\PY{o}{.}\PY{n}{min}\PY{p}{(}\PY{p}{)}\PY{p}{,}\PY{n}{wav}\PY{o}{.}\PY{n}{max}\PY{p}{(}\PY{p}{)}\PY{p}{]}\PY{p}{)}
        
        \PY{k}{for} \PY{n}{T} \PY{o+ow}{in} \PY{n+nb}{range}\PY{p}{(}\PY{l+m+mi}{8000}\PY{p}{,}\PY{l+m+mi}{5000}\PY{p}{,}\PY{o}{\PYZhy{}}\PY{l+m+mi}{200}\PY{p}{)}\PY{p}{:}
            \PY{k}{for} \PY{n}{i} \PY{o+ow}{in} \PY{n+nb}{range}\PY{p}{(}\PY{l+m+mi}{100}\PY{p}{)}\PY{p}{:}
                \PY{n}{b}\PY{p}{[}\PY{n}{i}\PY{p}{]} \PY{o}{=} \PY{n}{planck}\PY{p}{(}\PY{n}{T}\PY{p}{,}\PY{n}{wav}\PY{p}{[}\PY{n}{i}\PY{p}{]}\PY{o}{*}\PY{l+m+mf}{1e\PYZhy{}8}\PY{p}{)}
            \PY{n}{plt}\PY{o}{.}\PY{n}{plot}\PY{p}{(}\PY{n}{wav}\PY{p}{,}\PY{n}{b}\PY{p}{)}
\end{Verbatim}

    \begin{center}
    \adjustimage{max size={0.9\linewidth}{0.9\paperheight}}{Lab3_files/Lab3_7_0.png}
    \end{center}
    { \hspace*{\fill} \\}
    
    \textbf{Study the Planck function properties. B(T) increases at any
wavelength with the temperature, but much faster (exponentially, Wien
regime) at short wavelengths then at long wavelengths (linearly,
Rayleigh-Jeans regime). The peak divides the two regimes and shifts to
shorter wavelengths for higher temperature (Wien displacement law). The
spectrum-integrated Planck function (area under the curve in this linear
plot) increases steeply with temperature (Stefan-Boltzmann law).}

    \textbf{Add ,/ylog to the plot statement to make the y-axis logarithmic.
Inspect the result. Then make the x-axis also logarithmic and inspect
the result. Explain the slopes of the righthand part.}

    \begin{Verbatim}[commandchars=\\\{\}]
{\color{incolor}In [{\color{incolor}7}]:} \PY{n}{plt}\PY{o}{.}\PY{n}{figure}\PY{p}{(}\PY{p}{)}
        \PY{n}{plt}\PY{o}{.}\PY{n}{xlabel}\PY{p}{(}\PY{l+s}{r\PYZsq{}}\PY{l+s}{wavelength \PYZdl{}(}\PY{l+s}{\PYZbs{}}\PY{l+s}{AA)\PYZdl{}}\PY{l+s}{\PYZsq{}}\PY{p}{)}
        \PY{n}{plt}\PY{o}{.}\PY{n}{ylabel}\PY{p}{(}\PY{l+s}{\PYZsq{}}\PY{l+s}{Placnk function}\PY{l+s}{\PYZsq{}}\PY{p}{)}
        \PY{n}{plt}\PY{o}{.}\PY{n}{xlim}\PY{p}{(}\PY{p}{[}\PY{n}{wav}\PY{o}{.}\PY{n}{min}\PY{p}{(}\PY{p}{)}\PY{p}{,}\PY{n}{wav}\PY{o}{.}\PY{n}{max}\PY{p}{(}\PY{p}{)}\PY{p}{]}\PY{p}{)}
        
        \PY{k}{for} \PY{n}{T} \PY{o+ow}{in} \PY{n+nb}{range}\PY{p}{(}\PY{l+m+mi}{8000}\PY{p}{,}\PY{l+m+mi}{5000}\PY{p}{,}\PY{o}{\PYZhy{}}\PY{l+m+mi}{200}\PY{p}{)}\PY{p}{:}
            \PY{k}{for} \PY{n}{i} \PY{o+ow}{in} \PY{n+nb}{range}\PY{p}{(}\PY{l+m+mi}{100}\PY{p}{)}\PY{p}{:}
                \PY{n}{b}\PY{p}{[}\PY{n}{i}\PY{p}{]} \PY{o}{=} \PY{n}{planck}\PY{p}{(}\PY{n}{T}\PY{p}{,}\PY{n}{wav}\PY{p}{[}\PY{n}{i}\PY{p}{]}\PY{o}{*}\PY{l+m+mf}{1e\PYZhy{}8}\PY{p}{)}
            \PY{n}{plt}\PY{o}{.}\PY{n}{semilogy}\PY{p}{(}\PY{n}{wav}\PY{p}{,}\PY{n}{b}\PY{p}{)}
\end{Verbatim}

    \begin{center}
    \adjustimage{max size={0.9\linewidth}{0.9\paperheight}}{Lab3_files/Lab3_10_0.png}
    \end{center}
    { \hspace*{\fill} \\}
    
    \begin{Verbatim}[commandchars=\\\{\}]
{\color{incolor}In [{\color{incolor}8}]:} \PY{c}{\PYZsh{}plt.plot(wav,b)}
        \PY{n}{plt}\PY{o}{.}\PY{n}{xlabel}\PY{p}{(}\PY{l+s}{r\PYZsq{}}\PY{l+s}{wavelength \PYZdl{}(}\PY{l+s}{\PYZbs{}}\PY{l+s}{AA)\PYZdl{}}\PY{l+s}{\PYZsq{}}\PY{p}{)}
        \PY{n}{plt}\PY{o}{.}\PY{n}{ylabel}\PY{p}{(}\PY{l+s}{\PYZsq{}}\PY{l+s}{Placnk function}\PY{l+s}{\PYZsq{}}\PY{p}{)}
        \PY{n}{plt}\PY{o}{.}\PY{n}{xlim}\PY{p}{(}\PY{p}{[}\PY{n}{wav}\PY{o}{.}\PY{n}{min}\PY{p}{(}\PY{p}{)}\PY{p}{,}\PY{n}{wav}\PY{o}{.}\PY{n}{max}\PY{p}{(}\PY{p}{)}\PY{p}{]}\PY{p}{)}
        
        \PY{k}{for} \PY{n}{T} \PY{o+ow}{in} \PY{n+nb}{range}\PY{p}{(}\PY{l+m+mi}{8000}\PY{p}{,}\PY{l+m+mi}{5000}\PY{p}{,}\PY{o}{\PYZhy{}}\PY{l+m+mi}{200}\PY{p}{)}\PY{p}{:}
            \PY{k}{for} \PY{n}{i} \PY{o+ow}{in} \PY{n+nb}{range}\PY{p}{(}\PY{l+m+mi}{100}\PY{p}{)}\PY{p}{:}
                \PY{n}{b}\PY{p}{[}\PY{n}{i}\PY{p}{]} \PY{o}{=} \PY{n}{planck}\PY{p}{(}\PY{n}{T}\PY{p}{,}\PY{n}{wav}\PY{p}{[}\PY{n}{i}\PY{p}{]}\PY{o}{*}\PY{l+m+mf}{1e\PYZhy{}8}\PY{p}{)}
            \PY{n}{plt}\PY{o}{.}\PY{n}{loglog}\PY{p}{(}\PY{n}{wav}\PY{p}{,}\PY{n}{b}\PY{p}{)}
\end{Verbatim}

    \begin{center}
    \adjustimage{max size={0.9\linewidth}{0.9\paperheight}}{Lab3_files/Lab3_11_0.png}
    \end{center}
    { \hspace*{\fill} \\}
    
    The right hand part of the spectrum is the Raleigh-Jeans regime where
\(hc/\lambda >> kT\) and \(B(T) \rightarrow T/\lambda^{4}\) so in log
space it has a constant slope.

    \section*{3.2 Radiation through an isothermal
layer}\label{radiation-through-an-isothermal-layer}

    \textbf{Derive (11) from (10).}

    Since the layer is isothermal \(B_{\lambda}(T)\) is constant accross the
layer. Then eq. 10:

\[ I_{\lambda} = I_{\lambda}(0)e^{-\tau} + \int_{0}^{\tau}  B_{\lambda}[T(x)]e^{(\tau - \tau(x))} d\tau(x)\]

becomes:

\[ \int_{0}^{\tau} e^{(\tau - \tau(x))} d\tau(x) = 1 = e^{-\tau} \]
Thus:

\[ I_{\lambda} = I_{\lambda}(0)e^-{\tau} + B_{\lambda}(1 - e^{-\tau}) \]

    \textbf{Make plots of the emergent intesity \(I_{\lambda}\) for given
values of \(B_{\lambda}\) and \(I_{\lambda}(0)\) against \(\tau\):}

    \begin{Verbatim}[commandchars=\\\{\}]
{\color{incolor}In [{\color{incolor}9}]:} \PY{n}{plt}\PY{o}{.}\PY{n}{figure}\PY{p}{(}\PY{p}{)}
        \PY{n}{plt}\PY{o}{.}\PY{n}{xlabel}\PY{p}{(}\PY{l+s}{r\PYZsq{}}\PY{l+s}{\PYZdl{}}\PY{l+s}{\PYZbs{}}\PY{l+s}{tau\PYZdl{}}\PY{l+s}{\PYZsq{}}\PY{p}{)}
        \PY{n}{plt}\PY{o}{.}\PY{n}{ylabel}\PY{p}{(}\PY{l+s}{\PYZsq{}}\PY{l+s}{Intensity}\PY{l+s}{\PYZsq{}}\PY{p}{)}
        
        
        \PY{n}{B} \PY{o}{=} \PY{l+m+mf}{2.}
        \PY{n}{tau} \PY{o}{=} \PY{n}{np}\PY{o}{.}\PY{n}{arange}\PY{p}{(}\PY{l+m+mf}{0.01}\PY{p}{,}\PY{l+m+mi}{10}\PY{p}{,}\PY{l+m+mf}{0.01}\PY{p}{)} \PY{c}{\PYZsh{} set array tau = 0.01\PYZhy{}10 in steps 0.01}
        \PY{n}{integ} \PY{o}{=} \PY{n}{np}\PY{o}{.}\PY{n}{zeros\PYZus{}like}\PY{p}{(}\PY{n}{tau}\PY{p}{)} \PY{c}{\PYZsh{} declare float array of the same size}
        
        
        \PY{k}{for} \PY{n}{I0} \PY{o+ow}{in} \PY{n+nb}{range}\PY{p}{(}\PY{l+m+mi}{4}\PY{p}{,}\PY{o}{\PYZhy{}}\PY{l+m+mi}{1}\PY{p}{,}\PY{o}{\PYZhy{}}\PY{l+m+mi}{1}\PY{p}{)}\PY{p}{:} \PY{c}{\PYZsh{} step down from I0=4 to I0=0}
            \PY{k}{for} \PY{n}{i}\PY{p}{,}\PY{n}{t} \PY{o+ow}{in} \PY{n+nb}{enumerate}\PY{p}{(}\PY{n}{tau}\PY{p}{)}\PY{p}{:}
                \PY{n}{integ}\PY{p}{[}\PY{n}{i}\PY{p}{]}\PY{o}{=}\PY{n}{I0} \PY{o}{*} \PY{n}{np}\PY{o}{.}\PY{n}{exp}\PY{p}{(}\PY{o}{\PYZhy{}}\PY{n}{t}\PY{p}{)} \PY{o}{+} \PY{n}{B}\PY{o}{*}\PY{p}{(}\PY{l+m+mi}{1}\PY{o}{\PYZhy{}}\PY{n}{np}\PY{o}{.}\PY{n}{exp}\PY{p}{(}\PY{o}{\PYZhy{}}\PY{n}{t}\PY{p}{)}\PY{p}{)}
            \PY{k}{if} \PY{p}{(}\PY{n}{I0} \PY{o}{==} \PY{l+m+mi}{4}\PY{p}{)}\PY{p}{:}
                \PY{n}{plt}\PY{o}{.}\PY{n}{subplot}\PY{p}{(}\PY{l+m+mi}{121}\PY{p}{)}
                \PY{n}{plt}\PY{o}{.}\PY{n}{plot}\PY{p}{(}\PY{n}{tau}\PY{p}{,}\PY{n}{integ}\PY{p}{,} \PY{n}{label}\PY{o}{=}\PY{l+s}{r\PYZsq{}}\PY{l+s}{\PYZdl{}I\PYZus{}\PYZob{}}\PY{l+s}{\PYZbs{}}\PY{l+s}{lambda\PYZcb{}(0)\PYZdl{} = }\PY{l+s+si}{\PYZpc{}2d}\PY{l+s}{\PYZsq{}} \PY{o}{\PYZpc{}}\PY{k}{I0})
            \PY{k}{if} \PY{p}{(}\PY{n}{I0} \PY{o}{!=} \PY{l+m+mi}{4}\PY{p}{)}\PY{p}{:}
                \PY{n}{plt}\PY{o}{.}\PY{n}{subplot}\PY{p}{(}\PY{l+m+mi}{121}\PY{p}{)}
                \PY{n}{plt}\PY{o}{.}\PY{n}{plot}\PY{p}{(}\PY{n}{tau}\PY{p}{,}\PY{n}{integ}\PY{p}{,} \PY{n}{label}\PY{o}{=}\PY{l+s}{r\PYZsq{}}\PY{l+s}{\PYZdl{}I\PYZus{}\PYZob{}}\PY{l+s}{\PYZbs{}}\PY{l+s}{lambda\PYZcb{}(0)\PYZdl{} = }\PY{l+s+si}{\PYZpc{}2d}\PY{l+s}{\PYZsq{}} \PY{o}{\PYZpc{}}\PY{k}{I0})
        
        \PY{n}{plt}\PY{o}{.}\PY{n}{legend}\PY{p}{(}\PY{p}{)}        
        \PY{k}{for} \PY{n}{I0} \PY{o+ow}{in} \PY{n+nb}{range}\PY{p}{(}\PY{l+m+mi}{4}\PY{p}{,}\PY{o}{\PYZhy{}}\PY{l+m+mi}{1}\PY{p}{,}\PY{o}{\PYZhy{}}\PY{l+m+mi}{1}\PY{p}{)}\PY{p}{:} \PY{c}{\PYZsh{} step down from I0=4 to I0=0}
            \PY{k}{for} \PY{n}{i}\PY{p}{,}\PY{n}{t} \PY{o+ow}{in} \PY{n+nb}{enumerate}\PY{p}{(}\PY{n}{tau}\PY{p}{)}\PY{p}{:}
                \PY{n}{integ}\PY{p}{[}\PY{n}{i}\PY{p}{]}\PY{o}{=}\PY{n}{I0} \PY{o}{*} \PY{n}{np}\PY{o}{.}\PY{n}{exp}\PY{p}{(}\PY{o}{\PYZhy{}}\PY{n}{t}\PY{p}{)} \PY{o}{+} \PY{n}{B}\PY{o}{*}\PY{p}{(}\PY{l+m+mi}{1}\PY{o}{\PYZhy{}}\PY{n}{np}\PY{o}{.}\PY{n}{exp}\PY{p}{(}\PY{o}{\PYZhy{}}\PY{n}{t}\PY{p}{)}\PY{p}{)}
            \PY{k}{if} \PY{p}{(}\PY{n}{I0} \PY{o}{==} \PY{l+m+mi}{4}\PY{p}{)}\PY{p}{:}
                \PY{n}{plt}\PY{o}{.}\PY{n}{subplot}\PY{p}{(}\PY{l+m+mi}{122}\PY{p}{)}
                \PY{n}{plt}\PY{o}{.}\PY{n}{loglog}\PY{p}{(}\PY{n}{tau}\PY{p}{,}\PY{n}{integ}\PY{p}{,} \PY{n}{label}\PY{o}{=}\PY{l+s}{r\PYZsq{}}\PY{l+s}{\PYZdl{}I\PYZdl{} = }\PY{l+s+si}{\PYZpc{}2d}\PY{l+s}{\PYZsq{}} \PY{o}{\PYZpc{}}\PY{k}{I0})
            \PY{k}{if} \PY{p}{(}\PY{n}{I0} \PY{o}{!=} \PY{l+m+mi}{4}\PY{p}{)}\PY{p}{:}
                \PY{n}{plt}\PY{o}{.}\PY{n}{subplot}\PY{p}{(}\PY{l+m+mi}{122}\PY{p}{)}
                \PY{n}{plt}\PY{o}{.}\PY{n}{loglog}\PY{p}{(}\PY{n}{tau}\PY{p}{,}\PY{n}{integ}\PY{p}{,} \PY{n}{label}\PY{o}{=}\PY{l+s}{r\PYZsq{}}\PY{l+s}{\PYZdl{}I\PYZdl{} = }\PY{l+s+si}{\PYZpc{}2d}\PY{l+s}{\PYZsq{}} \PY{o}{\PYZpc{}}\PY{k}{I0})
\end{Verbatim}

    \begin{center}
    \adjustimage{max size={0.9\linewidth}{0.9\paperheight}}{Lab3_files/Lab3_17_0.png}
    \end{center}
    { \hspace*{\fill} \\}
    
    \textbf{How does \(I_{\lambda}\) depend on \(\tau\) for \(\tau \ll 1\)
when \(I_{\lambda}(0) = 0\) (add xlog ylog to study the behavior at
small \(\tau\))? And when \(I_{\lambda}(0) \gt B_{\lambda}\)? Such a
layer is called optically thin. Why?}

    If \(\tau \ll 1 I_{\lambda}\) grows with a slope of \(B_{\lambda}\).
When the incident flux is greater than the blackbody flux
\(I_{\lambda}\) is constant at small optical depths until and optical
depth of 1 where \(I_{\lambda}\) decreases to the blackbody.

    \textbf{A layer is called ``optically thick'' when it has
\(\tau \gg 1\). Why? The emergent intensity becomes independent of
\(\tau\) for large \(\tau\). Can you explain why this is so in physical
terms?}

    When \(\tau \gg 1\) the incident flux decays very quickly to the level
of the blackbody. This is because all of the incident flux has been
absorbed and is being reemitted in thermodynamic equalibrium with the
blackbody medium.

    \section*{3.3 Spectral lines from a solar reversing
layer}\label{spectral-lines-from-a-solar-reversing-layer}

    \textbf{Plot the Voigt function against u from u = -10 to u = +10 for a
= 0.1:}

    \textbf{Cursor back up and vary the value of a between a = 1 and a =
0.001 to see the effect of this parameter. Also add ,/ylog (without
setting yrange) to inspect the far wings of the profile. Use
approximation (18) to explain what you see.}

    \begin{Verbatim}[commandchars=\\\{\}]
{\color{incolor}In [{\color{incolor}10}]:} \PY{c}{\PYZsh{} function found at http://scipython.com/book/chapter\PYZhy{}8\PYZhy{}scipy/examples/the\PYZhy{}voigt\PYZhy{}profile/}
         
         \PY{k+kn}{from} \PY{n+nn}{scipy.special} \PY{k+kn}{import} \PY{n}{wofz}
         \PY{k}{def} \PY{n+nf}{V}\PY{p}{(}\PY{n}{x}\PY{p}{,} \PY{n}{alpha}\PY{p}{,} \PY{n}{gamma}\PY{p}{)}\PY{p}{:}
             \PY{l+s+sd}{\PYZdq{}\PYZdq{}\PYZdq{}}
         \PY{l+s+sd}{    Return the Voigt line shape at x with Lorentzian component HWHM gamma}
         \PY{l+s+sd}{    and Gaussian component HWHM alpha.}
         
         \PY{l+s+sd}{    \PYZdq{}\PYZdq{}\PYZdq{}}
             \PY{n}{sigma} \PY{o}{=} \PY{n}{alpha} \PY{o}{/} \PY{n}{np}\PY{o}{.}\PY{n}{sqrt}\PY{p}{(}\PY{l+m+mi}{2} \PY{o}{*} \PY{n}{np}\PY{o}{.}\PY{n}{log}\PY{p}{(}\PY{l+m+mi}{2}\PY{p}{)}\PY{p}{)}
         
             \PY{k}{return} \PY{n}{np}\PY{o}{.}\PY{n}{real}\PY{p}{(}\PY{n}{wofz}\PY{p}{(}\PY{p}{(}\PY{n}{x} \PY{o}{+} \PY{l+m+mi}{1j}\PY{o}{*}\PY{n}{gamma}\PY{p}{)}\PY{o}{/}\PY{n}{sigma}\PY{o}{/}\PY{n}{np}\PY{o}{.}\PY{n}{sqrt}\PY{p}{(}\PY{l+m+mi}{2}\PY{p}{)}\PY{p}{)}\PY{p}{)}\PY{o}{/}\PY{n}{sigma}\PY{o}{/}\PY{n}{np}\PY{o}{.}\PY{n}{sqrt}\PY{p}{(}\PY{l+m+mi}{2}\PY{o}{*}\PY{n}{np}\PY{o}{.}\PY{n}{pi}\PY{p}{)}
\end{Verbatim}

    \begin{Verbatim}[commandchars=\\\{\}]
{\color{incolor}In [{\color{incolor}11}]:} \PY{n}{alpha} \PY{o}{=} \PY{n}{np}\PY{o}{.}\PY{n}{linspace}\PY{p}{(}\PY{l+m+mf}{0.0001}\PY{p}{,}\PY{l+m+mi}{2}\PY{p}{,}\PY{l+m+mi}{3}\PY{p}{)}
         \PY{n}{gamma} \PY{o}{=} \PY{n}{np}\PY{o}{.}\PY{n}{linspace}\PY{p}{(}\PY{l+m+mf}{0.0001}\PY{p}{,}\PY{l+m+mi}{2}\PY{p}{,}\PY{l+m+mi}{3}\PY{p}{)}
         \PY{n}{u} \PY{o}{=} \PY{n}{np}\PY{o}{.}\PY{n}{linspace}\PY{p}{(}\PY{o}{\PYZhy{}}\PY{l+m+mi}{10}\PY{p}{,}\PY{l+m+mi}{10}\PY{p}{,}\PY{l+m+mi}{1000}\PY{p}{)}
         
         \PY{n}{plt}\PY{o}{.}\PY{n}{subplot}\PY{p}{(}\PY{l+m+mi}{121}\PY{p}{)}
         \PY{k}{for} \PY{n}{a} \PY{o+ow}{in} \PY{n}{alpha}\PY{p}{:}
             \PY{k}{for} \PY{n}{g} \PY{o+ow}{in} \PY{n}{gamma}\PY{p}{:}
                 \PY{n}{plt}\PY{o}{.}\PY{n}{plot}\PY{p}{(}\PY{n}{u}\PY{p}{,}\PY{n}{V}\PY{p}{(}\PY{n}{u}\PY{p}{,}\PY{n}{a}\PY{p}{,}\PY{n}{a}\PY{p}{)}\PY{p}{,} \PY{n}{label}\PY{o}{=}\PY{l+s}{\PYZdq{}}\PY{l+s}{a=\PYZob{}\PYZcb{} g=\PYZob{}\PYZcb{}}\PY{l+s}{\PYZdq{}}\PY{o}{.}\PY{n}{format}\PY{p}{(}\PY{n+nb}{int}\PY{p}{(}\PY{n}{a}\PY{p}{)}\PY{p}{,}\PY{n+nb}{int}\PY{p}{(}\PY{n}{g}\PY{p}{)}\PY{p}{)}\PY{p}{)}
         
         
         \PY{n}{plt}\PY{o}{.}\PY{n}{legend}\PY{p}{(}\PY{n}{prop}\PY{o}{=}\PY{p}{\PYZob{}}\PY{l+s}{\PYZsq{}}\PY{l+s}{size}\PY{l+s}{\PYZsq{}}\PY{p}{:}\PY{l+m+mi}{7}\PY{p}{\PYZcb{}}\PY{p}{)}
         \PY{n}{plt}\PY{o}{.}\PY{n}{subplot}\PY{p}{(}\PY{l+m+mi}{122}\PY{p}{)}    
         \PY{k}{for} \PY{n}{a} \PY{o+ow}{in} \PY{n}{alpha}\PY{p}{:}
             \PY{k}{for} \PY{n}{g} \PY{o+ow}{in} \PY{n}{gamma}\PY{p}{:}
                 \PY{n}{plt}\PY{o}{.}\PY{n}{semilogy}\PY{p}{(}\PY{n}{u}\PY{p}{,}\PY{n}{V}\PY{p}{(}\PY{n}{u}\PY{p}{,}\PY{n}{a}\PY{p}{,}\PY{n}{g}\PY{p}{)}\PY{p}{,} \PY{n}{label}\PY{o}{=}\PY{l+s}{\PYZdq{}}\PY{l+s}{a=\PYZob{}\PYZcb{} g=\PYZob{}\PYZcb{}}\PY{l+s}{\PYZdq{}}\PY{o}{.}\PY{n}{format}\PY{p}{(}\PY{n}{a}\PY{p}{,}\PY{n}{g}\PY{p}{)}\PY{p}{)}
\end{Verbatim}

    \begin{center}
    \adjustimage{max size={0.9\linewidth}{0.9\paperheight}}{Lab3_files/Lab3_26_0.png}
    \end{center}
    { \hspace*{\fill} \\}
    
    We can see that when \(\alpha\) is larger than \(\gamma\) the Voigt
function is more Gaussian and when \(\gamma\) is larger, it becomes a
Lorentzian

    \textbf{Now with the approximation in eq. 18 in order to be able to answer
questions better:}

    \begin{Verbatim}[commandchars=\\\{\}]
{\color{incolor}In [{\color{incolor}12}]:} \PY{k}{def} \PY{n+nf}{V2}\PY{p}{(}\PY{n}{a}\PY{p}{,}\PY{n}{u}\PY{p}{)}\PY{p}{:}
             \PY{c}{\PYZsh{}v = (np.exp(\PYZhy{}u**2) + a/(np.sqrt(np.pi)*u**2))/np.sqrt(np.pi)}
             \PY{n}{v} \PY{o}{=} \PY{p}{(}\PY{l+m+mf}{1.} \PY{o}{/} \PY{n}{np}\PY{o}{.}\PY{n}{sqrt}\PY{p}{(}\PY{n}{np}\PY{o}{.}\PY{n}{pi}\PY{p}{)}\PY{p}{)} \PY{o}{*} \PY{p}{(}\PY{n}{np}\PY{o}{.}\PY{n}{exp}\PY{p}{(}\PY{o}{\PYZhy{}}\PY{n}{u}\PY{o}{*}\PY{o}{*}\PY{l+m+mf}{2.}\PY{p}{)} \PY{o}{+} \PY{n}{a}\PY{o}{/}\PY{p}{(}\PY{n}{np}\PY{o}{.}\PY{n}{sqrt}\PY{p}{(}\PY{n}{np}\PY{o}{.}\PY{n}{pi}\PY{p}{)}\PY{o}{*}\PY{n}{u}\PY{o}{*}\PY{o}{*}\PY{l+m+mf}{2.}\PY{p}{)}\PY{p}{)}
             \PY{k}{return} \PY{n}{v}
         
         
         \PY{n}{alpha} \PY{o}{=} \PY{n}{np}\PY{o}{.}\PY{n}{linspace}\PY{p}{(}\PY{l+m+mf}{0.001}\PY{p}{,}\PY{l+m+mi}{1}\PY{p}{,}\PY{l+m+mi}{3}\PY{p}{)}
         \PY{c}{\PYZsh{}u = np.arange(201)/10. \PYZhy{} 10.}
         \PY{n}{u} \PY{o}{=} \PY{n}{np}\PY{o}{.}\PY{n}{linspace}\PY{p}{(}\PY{o}{\PYZhy{}}\PY{l+m+mi}{10}\PY{p}{,}\PY{l+m+mi}{10}\PY{p}{,}\PY{l+m+mi}{100}\PY{p}{)}
         
         \PY{n}{plt}\PY{o}{.}\PY{n}{subplot}\PY{p}{(}\PY{l+m+mi}{121}\PY{p}{)}
         \PY{k}{for} \PY{n}{a} \PY{o+ow}{in} \PY{n}{alpha}\PY{p}{:}
             \PY{n}{plt}\PY{o}{.}\PY{n}{plot}\PY{p}{(}\PY{n}{u}\PY{p}{,}\PY{n}{V2}\PY{p}{(}\PY{n}{a}\PY{p}{,}\PY{n}{np}\PY{o}{.}\PY{n}{abs}\PY{p}{(}\PY{n}{u}\PY{p}{)}\PY{p}{)}\PY{p}{,} \PY{n}{label}\PY{o}{=}\PY{l+s}{\PYZdq{}}\PY{l+s}{\PYZdl{}a=\PYZob{}\PYZcb{}\PYZdl{}}\PY{l+s}{\PYZdq{}}\PY{o}{.}\PY{n}{format}\PY{p}{(}\PY{n}{a}\PY{p}{)}\PY{p}{)}
         
         
         \PY{n}{plt}\PY{o}{.}\PY{n}{legend}\PY{p}{(}\PY{n}{prop}\PY{o}{=}\PY{p}{\PYZob{}}\PY{l+s}{\PYZsq{}}\PY{l+s}{size}\PY{l+s}{\PYZsq{}}\PY{p}{:}\PY{l+m+mi}{7}\PY{p}{\PYZcb{}}\PY{p}{)}
         \PY{n}{plt}\PY{o}{.}\PY{n}{subplot}\PY{p}{(}\PY{l+m+mi}{122}\PY{p}{)}    
         \PY{k}{for} \PY{n}{a} \PY{o+ow}{in} \PY{n}{alpha}\PY{p}{:}
             \PY{n}{plt}\PY{o}{.}\PY{n}{semilogy}\PY{p}{(}\PY{n}{u}\PY{p}{,}\PY{n}{V2}\PY{p}{(}\PY{n}{a}\PY{p}{,}\PY{n}{np}\PY{o}{.}\PY{n}{abs}\PY{p}{(}\PY{n}{u}\PY{p}{)}\PY{p}{)}\PY{p}{,} \PY{n}{label}\PY{o}{=}\PY{l+s}{\PYZdq{}}\PY{l+s}{a=\PYZob{}\PYZcb{}}\PY{l+s}{\PYZdq{}}\PY{o}{.}\PY{n}{format}\PY{p}{(}\PY{n}{a}\PY{p}{)}\PY{p}{)}
\end{Verbatim}

    \begin{center}
    \adjustimage{max size={0.9\linewidth}{0.9\paperheight}}{Lab3_files/Lab3_29_0.png}
    \end{center}
    { \hspace*{\fill} \\}
    
    We can see that as \(a\) is larger, the profile becomes more Lorentzian
(\(1/u^{2}\)). If \(a\) is smaller, the profile is more Gaussian.

    \subsection*{Emergent line profiles.}\label{emergent-line-profiles.}

\textbf{Write an IDL sequence that computes Schuster-Schwarzschild line
profiles. Take Tsurface = 5700 K, Tlayer = 4200 K, a =
0.1,\(\lambda = 5000\) Angstrom. These values are good choices for the
solar photosphere as seen in the optical part of the spectrum. First
plot a profile \(I\) against \(u\) for \(\tau(0) = 1\):}

\textbf{Study the appearance of the line in the spectrum as a function
of \(\tau0\) over the range log\(\tau(0)=-2 \rightarrow\)
log\(\tau(0) = 2\):}

    \begin{Verbatim}[commandchars=\\\{\}]
{\color{incolor}In [{\color{incolor}13}]:} \PY{n}{plt}\PY{o}{.}\PY{n}{figure}\PY{p}{(}\PY{p}{)}
         \PY{n}{Ts}\PY{o}{=}\PY{l+m+mi}{5700} \PY{c}{\PYZsh{} solar surface temperature}
         \PY{n}{Tl}\PY{o}{=}\PY{l+m+mi}{4200} \PY{c}{\PYZsh{} solar T\PYZhy{}min temperature = `reversing layer\PYZsq{}}
         \PY{n}{a}\PY{o}{=}\PY{l+m+mf}{0.1} \PY{c}{\PYZsh{} damping parameter}
         \PY{n}{wav}\PY{o}{=}\PY{l+m+mf}{5000e\PYZhy{}8} \PY{c}{\PYZsh{} wavelength in cm}
         \PY{n}{tau0}\PY{o}{=}\PY{l+m+mi}{1} \PY{c}{\PYZsh{} reversing layer thickness at line center}
         \PY{c}{\PYZsh{}u=np.arange(201)/10.\PYZhy{}10. \PYZsh{} u = \PYZhy{}10 to 10 in 0.1 steps}
         \PY{n}{u} \PY{o}{=} \PY{n}{np}\PY{o}{.}\PY{n}{linspace}\PY{p}{(}\PY{o}{\PYZhy{}}\PY{l+m+mi}{10}\PY{p}{,}\PY{l+m+mi}{10}\PY{p}{,}\PY{l+m+mi}{200}\PY{p}{)}
         \PY{n}{integ}\PY{o}{=}\PY{n}{np}\PY{o}{.}\PY{n}{zeros\PYZus{}like}\PY{p}{(}\PY{n}{u}\PY{p}{)} \PY{c}{\PYZsh{} declare array}
         \PY{k}{for} \PY{n}{i} \PY{o+ow}{in} \PY{n+nb}{range}\PY{p}{(}\PY{l+m+mi}{200}\PY{p}{)}\PY{p}{:}
             \PY{n}{tau}\PY{o}{=}\PY{n}{tau0} \PY{o}{*} \PY{n}{V2}\PY{p}{(}\PY{n}{a}\PY{p}{,}\PY{n+nb}{abs}\PY{p}{(}\PY{n}{u}\PY{p}{[}\PY{n}{i}\PY{p}{]}\PY{p}{)}\PY{p}{)}
             \PY{n}{integ}\PY{p}{[}\PY{n}{i}\PY{p}{]}\PY{o}{=}\PY{n}{planck}\PY{p}{(}\PY{n}{Ts}\PY{p}{,}\PY{n}{wav}\PY{p}{)} \PY{o}{*} \PY{n}{np}\PY{o}{.}\PY{n}{exp}\PY{p}{(}\PY{o}{\PYZhy{}}\PY{n}{tau}\PY{p}{)} \PY{o}{+} \PY{n}{planck}\PY{p}{(}\PY{n}{Tl}\PY{p}{,}\PY{n}{wav}\PY{p}{)}\PY{o}{*}\PY{p}{(}\PY{l+m+mf}{1.}\PY{o}{\PYZhy{}}\PY{n}{np}\PY{o}{.}\PY{n}{exp}\PY{p}{(}\PY{o}{\PYZhy{}}\PY{n}{tau}\PY{p}{)}\PY{p}{)}
         
         \PY{c}{\PYZsh{}plt.plot(u,integ)}
         
         \PY{n}{tau0}\PY{o}{=}\PY{p}{[}\PY{l+m+mf}{0.01}\PY{p}{,} \PY{l+m+mf}{0.05}\PY{p}{,} \PY{l+m+mf}{0.1}\PY{p}{,} \PY{l+m+mf}{0.5}\PY{p}{,} \PY{l+m+mi}{1}\PY{p}{,} \PY{l+m+mi}{5}\PY{p}{,} \PY{l+m+mi}{10}\PY{p}{,} \PY{l+m+mi}{50}\PY{p}{,} \PY{l+m+mi}{100}\PY{p}{]}
         \PY{k}{for} \PY{n}{t} \PY{o+ow}{in} \PY{n}{tau0}\PY{p}{:}
             \PY{k}{for} \PY{n}{i} \PY{o+ow}{in} \PY{n+nb}{range}\PY{p}{(}\PY{l+m+mi}{200}\PY{p}{)}\PY{p}{:}
                 \PY{n}{tau} \PY{o}{=} \PY{n}{t}\PY{o}{*}\PY{n}{V2}\PY{p}{(}\PY{n}{a}\PY{p}{,}\PY{n+nb}{abs}\PY{p}{(}\PY{n}{u}\PY{p}{[}\PY{n}{i}\PY{p}{]}\PY{p}{)}\PY{p}{)} 
                 \PY{n}{integ}\PY{p}{[}\PY{n}{i}\PY{p}{]}\PY{o}{=}\PY{n}{planck}\PY{p}{(}\PY{n}{Ts}\PY{p}{,}\PY{n}{wav}\PY{p}{)} \PY{o}{*} \PY{n}{np}\PY{o}{.}\PY{n}{exp}\PY{p}{(}\PY{o}{\PYZhy{}}\PY{n}{tau}\PY{p}{)} \PY{o}{+} \PY{n}{planck}\PY{p}{(}\PY{n}{Tl}\PY{p}{,}\PY{n}{wav}\PY{p}{)}\PY{o}{*}\PY{p}{(}\PY{l+m+mf}{1.}\PY{o}{\PYZhy{}}\PY{n}{np}\PY{o}{.}\PY{n}{exp}\PY{p}{(}\PY{o}{\PYZhy{}}\PY{n}{tau}\PY{p}{)}\PY{p}{)}
             \PY{n}{plt}\PY{o}{.}\PY{n}{plot}\PY{p}{(}\PY{n}{u}\PY{p}{,}\PY{n}{integ}\PY{p}{,} \PY{n}{label}\PY{o}{=}\PY{l+s}{\PYZsq{}}\PY{l+s}{\PYZdl{}}\PY{l+s+se}{\PYZbs{}\PYZbs{}}\PY{l+s}{tau = \PYZob{}\PYZcb{}\PYZdl{}}\PY{l+s}{\PYZsq{}}\PY{o}{.}\PY{n}{format}\PY{p}{(}\PY{n}{t}\PY{p}{)}\PY{p}{)}
         
         \PY{n}{plt}\PY{o}{.}\PY{n}{legend}\PY{p}{(}\PY{p}{)}
         \PY{n}{plt}\PY{o}{.}\PY{n}{show}\PY{p}{(}\PY{p}{)}
\end{Verbatim}

    \begin{center}
    \adjustimage{max size={0.9\linewidth}{0.9\paperheight}}{Lab3_files/Lab3_32_0.png}
    \end{center}
    { \hspace*{\fill} \\}
    
    \textbf{How do you explain the profile shapes for \(\tau \ll 1\)?}

The profiles are much thinner and are have small wings and are gaussian
dominated.

\textbf{Why is there a low-intensity saturation limit for
\(\tau \gg 1\)?} The intensity drops to the black body intensity of the
cooler layer (not surface) at and around line center.

\textbf{Why do the line wings develop only for very large \(\tau(0)\)?}
As \(\tau\) increases, the line saturates and doesnt develope wings
until the optical depth is high enough that the density is high enough
that other processes like pressure broadening take over.

\textbf{Where do the wings end?} They go on forever but diminish
asymptotically.

\textbf{For which values of \(\tau(0)\) is the layer optically thin,
respectively optically thick, at line center? And at \(u = 5\)?} For
values \(\gt \tau(0) \approx 1\) the layer becomes optically thick at
linecenter. The wings become optically thick when there is a deviation
from the continuum so \(\tau(0) \approx 50\)

\textbf{Now study the dependence of these line profiles on wavelength by
repeating the above for \(\lambda = 2000\) Angstroms (ultraviolet) and
\(\lambda = 10000\) Angstroms (near infrared). What sets the top value
\(I_{cont}\) and the limit value reached at line center by \(I(0)\)?}

\(I_{cont}\) is set by the blackbody emission from the surface while
\(I(0)\) at line center is set by the blackbody of the layer.

    \begin{Verbatim}[commandchars=\\\{\}]
{\color{incolor}In [{\color{incolor}14}]:} \PY{n}{Ts}\PY{o}{=}\PY{l+m+mi}{5700} \PY{c}{\PYZsh{} solar surface temperature}
         \PY{n}{Tl}\PY{o}{=}\PY{l+m+mi}{4200} \PY{c}{\PYZsh{} solar T\PYZhy{}min temperature = `reversing layer\PYZsq{}}
         \PY{n}{a}\PY{o}{=}\PY{l+m+mf}{0.1} \PY{c}{\PYZsh{} damping parameter}
         \PY{n}{wav}\PY{o}{=}\PY{l+m+mf}{2000e\PYZhy{}8} \PY{c}{\PYZsh{} wavelength in cm}
         \PY{n}{tau0}\PY{o}{=}\PY{l+m+mi}{1} \PY{c}{\PYZsh{} reversing layer thickness at line center}
         \PY{c}{\PYZsh{}u=np.arange(201)/10.\PYZhy{}10. \PYZsh{} u = \PYZhy{}10 to 10 in 0.1 steps}
         \PY{n}{u} \PY{o}{=} \PY{n}{np}\PY{o}{.}\PY{n}{linspace}\PY{p}{(}\PY{o}{\PYZhy{}}\PY{l+m+mi}{10}\PY{p}{,}\PY{l+m+mi}{10}\PY{p}{,}\PY{l+m+mi}{200}\PY{p}{)}
         \PY{n}{integ}\PY{o}{=}\PY{n}{np}\PY{o}{.}\PY{n}{zeros\PYZus{}like}\PY{p}{(}\PY{n}{u}\PY{p}{)} \PY{c}{\PYZsh{} declare array}
         \PY{k}{for} \PY{n}{i} \PY{o+ow}{in} \PY{n+nb}{range}\PY{p}{(}\PY{l+m+mi}{200}\PY{p}{)}\PY{p}{:}
             \PY{n}{tau}\PY{o}{=}\PY{n}{tau0} \PY{o}{*} \PY{n}{V2}\PY{p}{(}\PY{n}{a}\PY{p}{,}\PY{n+nb}{abs}\PY{p}{(}\PY{n}{u}\PY{p}{[}\PY{n}{i}\PY{p}{]}\PY{p}{)}\PY{p}{)}
             \PY{n}{integ}\PY{p}{[}\PY{n}{i}\PY{p}{]}\PY{o}{=}\PY{n}{planck}\PY{p}{(}\PY{n}{Ts}\PY{p}{,}\PY{n}{wav}\PY{p}{)} \PY{o}{*} \PY{n}{np}\PY{o}{.}\PY{n}{exp}\PY{p}{(}\PY{o}{\PYZhy{}}\PY{n}{tau}\PY{p}{)} \PY{o}{+} \PY{n}{planck}\PY{p}{(}\PY{n}{Tl}\PY{p}{,}\PY{n}{wav}\PY{p}{)}\PY{o}{*}\PY{p}{(}\PY{l+m+mf}{1.}\PY{o}{\PYZhy{}}\PY{n}{np}\PY{o}{.}\PY{n}{exp}\PY{p}{(}\PY{o}{\PYZhy{}}\PY{n}{tau}\PY{p}{)}\PY{p}{)}
         
         \PY{c}{\PYZsh{}plt.plot(u,integ)}
         
         \PY{n}{tau0}\PY{o}{=}\PY{p}{[}\PY{l+m+mf}{0.01}\PY{p}{,} \PY{l+m+mf}{0.05}\PY{p}{,} \PY{l+m+mf}{0.1}\PY{p}{,} \PY{l+m+mf}{0.5}\PY{p}{,} \PY{l+m+mi}{1}\PY{p}{,} \PY{l+m+mi}{5}\PY{p}{,} \PY{l+m+mi}{10}\PY{p}{,} \PY{l+m+mi}{50}\PY{p}{,} \PY{l+m+mi}{100}\PY{p}{]}
         \PY{k}{for} \PY{n}{t} \PY{o+ow}{in} \PY{n}{tau0}\PY{p}{:}
             \PY{k}{for} \PY{n}{i} \PY{o+ow}{in} \PY{n+nb}{range}\PY{p}{(}\PY{l+m+mi}{200}\PY{p}{)}\PY{p}{:}
                 \PY{n}{tau} \PY{o}{=} \PY{n}{t}\PY{o}{*}\PY{n}{V2}\PY{p}{(}\PY{n}{a}\PY{p}{,}\PY{n+nb}{abs}\PY{p}{(}\PY{n}{u}\PY{p}{[}\PY{n}{i}\PY{p}{]}\PY{p}{)}\PY{p}{)} 
                 \PY{n}{integ}\PY{p}{[}\PY{n}{i}\PY{p}{]}\PY{o}{=}\PY{n}{planck}\PY{p}{(}\PY{n}{Ts}\PY{p}{,}\PY{n}{wav}\PY{p}{)} \PY{o}{*} \PY{n}{np}\PY{o}{.}\PY{n}{exp}\PY{p}{(}\PY{o}{\PYZhy{}}\PY{n}{tau}\PY{p}{)} \PY{o}{+} \PY{n}{planck}\PY{p}{(}\PY{n}{Tl}\PY{p}{,}\PY{n}{wav}\PY{p}{)}\PY{o}{*}\PY{p}{(}\PY{l+m+mf}{1.}\PY{o}{\PYZhy{}}\PY{n}{np}\PY{o}{.}\PY{n}{exp}\PY{p}{(}\PY{o}{\PYZhy{}}\PY{n}{tau}\PY{p}{)}\PY{p}{)}
             \PY{n}{plt}\PY{o}{.}\PY{n}{subplot}\PY{p}{(}\PY{l+m+mi}{121}\PY{p}{)}
             \PY{n}{plt}\PY{o}{.}\PY{n}{plot}\PY{p}{(}\PY{n}{u}\PY{p}{,}\PY{n}{integ}\PY{p}{,} \PY{n}{label}\PY{o}{=}\PY{l+s}{\PYZsq{}}\PY{l+s}{\PYZdl{}}\PY{l+s+se}{\PYZbs{}\PYZbs{}}\PY{l+s}{tau = \PYZob{}\PYZcb{}\PYZdl{}}\PY{l+s}{\PYZsq{}}\PY{o}{.}\PY{n}{format}\PY{p}{(}\PY{n}{t}\PY{p}{)}\PY{p}{)}
         \PY{n}{wav}\PY{o}{=}\PY{l+m+mf}{10000e\PYZhy{}8} \PY{c}{\PYZsh{} wavelength in cm}
         \PY{k}{for} \PY{n}{t} \PY{o+ow}{in} \PY{n}{tau0}\PY{p}{:}
             \PY{k}{for} \PY{n}{i} \PY{o+ow}{in} \PY{n+nb}{range}\PY{p}{(}\PY{l+m+mi}{200}\PY{p}{)}\PY{p}{:}
                 \PY{n}{tau} \PY{o}{=} \PY{n}{t}\PY{o}{*}\PY{n}{V2}\PY{p}{(}\PY{n}{a}\PY{p}{,}\PY{n+nb}{abs}\PY{p}{(}\PY{n}{u}\PY{p}{[}\PY{n}{i}\PY{p}{]}\PY{p}{)}\PY{p}{)} 
                 \PY{n}{integ}\PY{p}{[}\PY{n}{i}\PY{p}{]}\PY{o}{=}\PY{n}{planck}\PY{p}{(}\PY{n}{Ts}\PY{p}{,}\PY{n}{wav}\PY{p}{)} \PY{o}{*} \PY{n}{np}\PY{o}{.}\PY{n}{exp}\PY{p}{(}\PY{o}{\PYZhy{}}\PY{n}{tau}\PY{p}{)} \PY{o}{+} \PY{n}{planck}\PY{p}{(}\PY{n}{Tl}\PY{p}{,}\PY{n}{wav}\PY{p}{)}\PY{o}{*}\PY{p}{(}\PY{l+m+mf}{1.}\PY{o}{\PYZhy{}}\PY{n}{np}\PY{o}{.}\PY{n}{exp}\PY{p}{(}\PY{o}{\PYZhy{}}\PY{n}{tau}\PY{p}{)}\PY{p}{)}
             \PY{n}{plt}\PY{o}{.}\PY{n}{subplot}\PY{p}{(}\PY{l+m+mi}{122}\PY{p}{)}
             \PY{n}{plt}\PY{o}{.}\PY{n}{plot}\PY{p}{(}\PY{n}{u}\PY{p}{,}\PY{n}{integ}\PY{p}{,} \PY{n}{label}\PY{o}{=}\PY{l+s}{\PYZsq{}}\PY{l+s}{\PYZdl{}}\PY{l+s+se}{\PYZbs{}\PYZbs{}}\PY{l+s}{tau = \PYZob{}\PYZcb{}\PYZdl{}}\PY{l+s}{\PYZsq{}}\PY{o}{.}\PY{n}{format}\PY{p}{(}\PY{n}{t}\PY{p}{)}\PY{p}{)}
\end{Verbatim}

    \begin{center}
    \adjustimage{max size={0.9\linewidth}{0.9\paperheight}}{Lab3_files/Lab3_34_0.png}
    \end{center}
    { \hspace*{\fill} \\}
    
    In the ultraviolet region (2000 Angstroms) the continuum emission is
about 10 times less than in the infrared (10000 Angstroms).

\textbf{Check these values by computing them directly on the command
line. What happens to these values at other wavelengths?}

    \begin{Verbatim}[commandchars=\\\{\}]
{\color{incolor}In [{\color{incolor}24}]:} \PY{n}{wav} \PY{o}{=} \PY{l+m+mf}{2000e\PYZhy{}8}
         \PY{n}{p} \PY{o}{=} \PY{n}{planck}\PY{p}{(}\PY{n}{Ts}\PY{p}{,}\PY{n}{wav}\PY{p}{)} \PY{o}{*} \PY{n}{np}\PY{o}{.}\PY{n}{exp}\PY{p}{(}\PY{o}{\PYZhy{}}\PY{n}{tau}\PY{p}{)} \PY{o}{+} \PY{n}{planck}\PY{p}{(}\PY{n}{Tl}\PY{p}{,}\PY{n}{wav}\PY{p}{)}\PY{o}{*}\PY{p}{(}\PY{l+m+mf}{1.}\PY{o}{\PYZhy{}}\PY{n}{np}\PY{o}{.}\PY{n}{exp}\PY{p}{(}\PY{o}{\PYZhy{}}\PY{n}{tau}\PY{p}{)}\PY{p}{)}
         \PY{k}{print} \PY{l+s}{\PYZsq{}}\PY{l+s}{I = \PYZob{}:.2\PYZcb{} at \PYZob{}\PYZcb{} Angstroms}\PY{l+s}{\PYZsq{}}\PY{o}{.}\PY{n}{format}\PY{p}{(}\PY{n}{p}\PY{p}{,}\PY{n}{wav}\PY{p}{)}
         
         \PY{n}{wav} \PY{o}{=} \PY{l+m+mf}{10000e\PYZhy{}8}
         \PY{n}{p} \PY{o}{=} \PY{n}{planck}\PY{p}{(}\PY{n}{Ts}\PY{p}{,}\PY{n}{wav}\PY{p}{)} \PY{o}{*} \PY{n}{np}\PY{o}{.}\PY{n}{exp}\PY{p}{(}\PY{o}{\PYZhy{}}\PY{n}{tau}\PY{p}{)} \PY{o}{+} \PY{n}{planck}\PY{p}{(}\PY{n}{Tl}\PY{p}{,}\PY{n}{wav}\PY{p}{)}\PY{o}{*}\PY{p}{(}\PY{l+m+mf}{1.}\PY{o}{\PYZhy{}}\PY{n}{np}\PY{o}{.}\PY{n}{exp}\PY{p}{(}\PY{o}{\PYZhy{}}\PY{n}{tau}\PY{p}{)}\PY{p}{)}
         \PY{k}{print} \PY{l+s}{\PYZsq{}}\PY{l+s}{I = \PYZob{}:.2\PYZcb{} at \PYZob{}\PYZcb{} Angstroms}\PY{l+s}{\PYZsq{}}\PY{o}{.}\PY{n}{format}\PY{p}{(}\PY{n}{p}\PY{p}{,}\PY{n}{wav}\PY{p}{)}
         
         \PY{n}{wav} \PY{o}{=} \PY{l+m+mf}{1000e\PYZhy{}8}
         \PY{n}{p} \PY{o}{=} \PY{n}{planck}\PY{p}{(}\PY{n}{Ts}\PY{p}{,}\PY{n}{wav}\PY{p}{)} \PY{o}{*} \PY{n}{np}\PY{o}{.}\PY{n}{exp}\PY{p}{(}\PY{o}{\PYZhy{}}\PY{n}{tau}\PY{p}{)} \PY{o}{+} \PY{n}{planck}\PY{p}{(}\PY{n}{Tl}\PY{p}{,}\PY{n}{wav}\PY{p}{)}\PY{o}{*}\PY{p}{(}\PY{l+m+mf}{1.}\PY{o}{\PYZhy{}}\PY{n}{np}\PY{o}{.}\PY{n}{exp}\PY{p}{(}\PY{o}{\PYZhy{}}\PY{n}{tau}\PY{p}{)}\PY{p}{)}
         \PY{k}{print} \PY{l+s}{\PYZsq{}}\PY{l+s}{I = \PYZob{}:.2\PYZcb{} at \PYZob{}\PYZcb{} Angstroms}\PY{l+s}{\PYZsq{}}\PY{o}{.}\PY{n}{format}\PY{p}{(}\PY{n}{p}\PY{p}{,}\PY{n}{wav}\PY{p}{)}
         
         \PY{n}{wav} \PY{o}{=} \PY{l+m+mf}{20000e\PYZhy{}8} 
         \PY{n}{p} \PY{o}{=} \PY{n}{planck}\PY{p}{(}\PY{n}{Ts}\PY{p}{,}\PY{n}{wav}\PY{p}{)} \PY{o}{*} \PY{n}{np}\PY{o}{.}\PY{n}{exp}\PY{p}{(}\PY{o}{\PYZhy{}}\PY{n}{tau}\PY{p}{)} \PY{o}{+} \PY{n}{planck}\PY{p}{(}\PY{n}{Tl}\PY{p}{,}\PY{n}{wav}\PY{p}{)}\PY{o}{*}\PY{p}{(}\PY{l+m+mf}{1.}\PY{o}{\PYZhy{}}\PY{n}{np}\PY{o}{.}\PY{n}{exp}\PY{p}{(}\PY{o}{\PYZhy{}}\PY{n}{tau}\PY{p}{)}\PY{p}{)}
         \PY{k}{print} \PY{l+s}{\PYZsq{}}\PY{l+s}{I = \PYZob{}:.2\PYZcb{} at \PYZob{}\PYZcb{} Angstroms}\PY{l+s}{\PYZsq{}}\PY{o}{.}\PY{n}{format}\PY{p}{(}\PY{n}{p}\PY{p}{,}\PY{n}{wav}\PY{p}{)}
         
         \PY{n}{wav} \PY{o}{=} \PY{n}{np}\PY{o}{.}\PY{n}{linspace}\PY{p}{(}\PY{l+m+mf}{2000e\PYZhy{}8}\PY{p}{,}\PY{l+m+mf}{10000e\PYZhy{}8}\PY{p}{)}
         \PY{n}{plt}\PY{o}{.}\PY{n}{plot}\PY{p}{(}\PY{n}{wav}\PY{p}{,}\PY{n}{planck}\PY{p}{(}\PY{n}{Ts}\PY{p}{,}\PY{n}{wav}\PY{p}{)} \PY{o}{*} \PY{n}{np}\PY{o}{.}\PY{n}{exp}\PY{p}{(}\PY{o}{\PYZhy{}}\PY{n}{tau}\PY{p}{)} \PY{o}{+} \PY{n}{planck}\PY{p}{(}\PY{n}{Tl}\PY{p}{,}\PY{n}{wav}\PY{p}{)}\PY{o}{*}\PY{p}{(}\PY{l+m+mf}{1.}\PY{o}{\PYZhy{}}\PY{n}{np}\PY{o}{.}\PY{n}{exp}\PY{p}{(}\PY{o}{\PYZhy{}}\PY{n}{tau}\PY{p}{)}\PY{p}{)}\PY{p}{)}
         \PY{n}{plt}\PY{o}{.}\PY{n}{show}\PY{p}{(}\PY{p}{)}
\end{Verbatim}

    \begin{Verbatim}[commandchars=\\\{\}]
I = 1e+14 at 2e-05 Angstroms
I = 1.1e+14 at 0.0001 Angstroms
I = 2.1e+12 at 1e-05 Angstroms
I = 1.5e+13 at 0.0002 Angstroms
    \end{Verbatim}

    \begin{center}
    \adjustimage{max size={0.9\linewidth}{0.9\paperheight}}{Lab3_files/Lab3_36_1.png}
    \end{center}
    { \hspace*{\fill} \\}
    
    At other wavelengths it traces out the Blackbody with
\(T = T_{surface}\)

    \textbf{Observed spectra that are measured in detector counts without
absolute intensity calibration (as in your Clea-Spec data gathering in
Exercise 1) are usually scaled to the local continuum intensity by
plotting \(I_{\lambda}/I_{cont}\) against wavelength. Do that for the
above profiles at the same three wavelengths:}

    \begin{Verbatim}[commandchars=\\\{\}]
{\color{incolor}In [{\color{incolor}16}]:} \PY{n}{waves} \PY{o}{=} \PY{n}{np}\PY{o}{.}\PY{n}{array}\PY{p}{(}\PY{p}{[}\PY{l+m+mi}{2000}\PY{p}{,} \PY{l+m+mi}{5000}\PY{p}{,} \PY{l+m+mi}{10000}\PY{p}{]}\PY{p}{)}\PY{o}{*}\PY{l+m+mf}{1e\PYZhy{}8} \PY{c}{\PYZsh{} Angstrom}
         \PY{k}{for} \PY{n}{wav} \PY{o+ow}{in} \PY{n}{waves}\PY{p}{:}
             \PY{k}{for} \PY{n}{t} \PY{o+ow}{in} \PY{n}{tau0}\PY{p}{:}
                 \PY{k}{for} \PY{n}{i} \PY{o+ow}{in} \PY{n+nb}{range}\PY{p}{(}\PY{l+m+mi}{200}\PY{p}{)}\PY{p}{:}
                     \PY{n}{tau} \PY{o}{=} \PY{n}{t}\PY{o}{*}\PY{n}{V2}\PY{p}{(}\PY{n}{a}\PY{p}{,}\PY{n+nb}{abs}\PY{p}{(}\PY{n}{u}\PY{p}{[}\PY{n}{i}\PY{p}{]}\PY{p}{)}\PY{p}{)} 
                     \PY{n}{integ}\PY{p}{[}\PY{n}{i}\PY{p}{]}\PY{o}{=}\PY{n}{planck}\PY{p}{(}\PY{n}{Ts}\PY{p}{,}\PY{n}{wav}\PY{p}{)} \PY{o}{*} \PY{n}{np}\PY{o}{.}\PY{n}{exp}\PY{p}{(}\PY{o}{\PYZhy{}}\PY{n}{tau}\PY{p}{)} \PY{o}{+} \PY{n}{planck}\PY{p}{(}\PY{n}{Tl}\PY{p}{,}\PY{n}{wav}\PY{p}{)}\PY{o}{*}\PY{p}{(}\PY{l+m+mf}{1.}\PY{o}{\PYZhy{}}\PY{n}{np}\PY{o}{.}\PY{n}{exp}\PY{p}{(}\PY{o}{\PYZhy{}}\PY{n}{tau}\PY{p}{)}\PY{p}{)}
                 \PY{n}{integ}\PY{o}{=}\PY{n}{integ}\PY{o}{/}\PY{n}{planck}\PY{p}{(}\PY{n}{Ts}\PY{p}{,}\PY{n}{wav}\PY{p}{)} \PY{c}{\PYZsh{} convert into relative intensity}
                 \PY{n}{plt}\PY{o}{.}\PY{n}{plot}\PY{p}{(}\PY{n}{u}\PY{p}{,}\PY{n}{integ}\PY{p}{)}
\end{Verbatim}

    \begin{center}
    \adjustimage{max size={0.9\linewidth}{0.9\paperheight}}{Lab3_files/Lab3_39_0.png}
    \end{center}
    { \hspace*{\fill} \\}
    
    \textbf{Explain the wavelength dependencies in this plot.}

The profile depth is larger where the difference between the two blackbody curves is largest. The line depth is smallest where the blackbody curves are closer to each other. 

    \section*{3.4 The equivalent width of spectral
lines}\label{the-equivalent-width-of-spectral-lines}

    \begin{Verbatim}[commandchars=\\\{\}]
{\color{incolor}In [{\color{incolor}17}]:} \PY{n}{wav}\PY{o}{=}\PY{l+m+mf}{5000.E\PYZhy{}8}
         \PY{n}{Ts}\PY{o}{=}\PY{l+m+mi}{5700}
         \PY{n}{Tl}\PY{o}{=}\PY{l+m+mi}{4200}
         \PY{k}{def} \PY{n+nf}{profile}\PY{p}{(}\PY{n}{a}\PY{p}{,}\PY{n}{tau0}\PY{p}{,}\PY{n}{u}\PY{p}{)}\PY{p}{:}
             \PY{l+s+sd}{\PYZdq{}\PYZdq{}\PYZdq{} }
         \PY{l+s+sd}{    return a Schuster\PYZhy{}Schwarzschild profile}
         \PY{l+s+sd}{    input: a = damping parameter}
         \PY{l+s+sd}{        tau0 = SS layer thickness at line center}
         \PY{l+s+sd}{        u = wavelength array in Doppler units}
         \PY{l+s+sd}{    output: int = intensity array}
         \PY{l+s+sd}{    \PYZdq{}\PYZdq{}\PYZdq{}}
             
             \PY{n}{integ}\PY{o}{=}\PY{n}{np}\PY{o}{.}\PY{n}{zeros\PYZus{}like}\PY{p}{(}\PY{n}{u}\PY{p}{)}
             \PY{n}{usize}\PY{o}{=}\PY{n+nb}{len}\PY{p}{(}\PY{n}{u}\PY{p}{)} \PY{c}{\PYZsh{} IDL SIZE returns array type and dimensions}
             \PY{k}{for} \PY{n}{i} \PY{o+ow}{in} \PY{n+nb}{range}\PY{p}{(}\PY{n}{usize}\PY{p}{)}\PY{p}{:}
                 \PY{n}{tau}\PY{o}{=}\PY{n}{tau0} \PY{o}{*} \PY{n}{V2}\PY{p}{(}\PY{n}{a}\PY{p}{,}\PY{n+nb}{abs}\PY{p}{(}\PY{n}{u}\PY{p}{[}\PY{n}{i}\PY{p}{]}\PY{p}{)}\PY{p}{)}
                 \PY{n}{integ}\PY{p}{[}\PY{n}{i}\PY{p}{]}\PY{o}{=}\PY{n}{planck}\PY{p}{(}\PY{n}{Ts}\PY{p}{,}\PY{n}{wav}\PY{p}{)} \PY{o}{*} \PY{n}{np}\PY{o}{.}\PY{n}{exp}\PY{p}{(}\PY{o}{\PYZhy{}}\PY{n}{tau}\PY{p}{)} \PY{o}{+} \PY{n}{planck}\PY{p}{(}\PY{n}{Tl}\PY{p}{,}\PY{n}{wav}\PY{p}{)}\PY{o}{*}\PY{p}{(}\PY{l+m+mi}{1}\PY{o}{\PYZhy{}}\PY{n}{np}\PY{o}{.}\PY{n}{exp}\PY{p}{(}\PY{o}{\PYZhy{}}\PY{n}{tau}\PY{p}{)}\PY{p}{)}
             \PY{k}{return} \PY{n}{integ}
\end{Verbatim}

    \textbf{Check your routine:}

    \begin{Verbatim}[commandchars=\\\{\}]
{\color{incolor}In [{\color{incolor}18}]:} \PY{n}{Ts}\PY{o}{=}\PY{l+m+mi}{5700}
         \PY{n}{Tl}\PY{o}{=}\PY{l+m+mi}{4200}
         \PY{n}{u} \PY{o}{=} \PY{n}{np}\PY{o}{.}\PY{n}{arange}\PY{p}{(}\PY{o}{\PYZhy{}}\PY{l+m+mi}{200}\PY{p}{,}\PY{l+m+mi}{200}\PY{p}{,}\PY{l+m+mf}{0.4}\PY{p}{)}
         \PY{n}{a} \PY{o}{=} \PY{l+m+mf}{0.1}
         \PY{n}{tau0} \PY{o}{=} \PY{l+m+mf}{1e2}
         \PY{n}{integ} \PY{o}{=} \PY{n}{profile}\PY{p}{(}\PY{n}{a}\PY{p}{,}\PY{n}{tau0}\PY{p}{,}\PY{n}{u}\PY{p}{)}
         \PY{n}{plt}\PY{o}{.}\PY{n}{plot}\PY{p}{(}\PY{n}{u}\PY{p}{,}\PY{n}{integ}\PY{p}{)}
         \PY{n}{plt}\PY{o}{.}\PY{n}{show}\PY{p}{(}\PY{p}{)}
\end{Verbatim}

    \begin{center}
    \adjustimage{max size={0.9\linewidth}{0.9\paperheight}}{Lab3_files/Lab3_44_0.png}
    \end{center}
    { \hspace*{\fill} \\}
    
    \textbf{Continue by computing the equivalent width with the IDL TOTAL
function (same as numpy sum()):}

    \begin{Verbatim}[commandchars=\\\{\}]
{\color{incolor}In [{\color{incolor}19}]:} \PY{n}{reldepth} \PY{o}{=} \PY{p}{(}\PY{n}{integ}\PY{p}{[}\PY{l+m+mi}{0}\PY{p}{]} \PY{o}{\PYZhy{}} \PY{n}{integ}\PY{p}{)}\PY{o}{/}\PY{n}{integ}\PY{p}{[}\PY{l+m+mi}{0}\PY{p}{]}
         \PY{n}{plt}\PY{o}{.}\PY{n}{plot}\PY{p}{(}\PY{n}{u}\PY{p}{,}\PY{n}{reldepth}\PY{p}{)}
         \PY{n}{eqw} \PY{o}{=} \PY{n}{np}\PY{o}{.}\PY{n}{sum}\PY{p}{(}\PY{n}{reldepth}\PY{p}{)}\PY{o}{*}\PY{l+m+mf}{0.4}
         \PY{k}{print} \PY{l+s}{\PYZsq{}}\PY{l+s}{equivalent width = \PYZob{}:.2\PYZcb{}}\PY{l+s}{\PYZsq{}}\PY{o}{.}\PY{n}{format}\PY{p}{(}\PY{n}{eqw}\PY{p}{)}
\end{Verbatim}

    \begin{Verbatim}[commandchars=\\\{\}]
equivalent width = 5.8
    \end{Verbatim}

    \begin{center}
    \adjustimage{max size={0.9\linewidth}{0.9\paperheight}}{Lab3_files/Lab3_46_1.png}
    \end{center}
    { \hspace*{\fill} \\}
    
    \section*{3.5 The curve of growth}\label{the-curve-of-growth}

\textbf{Compute and plot a curve of growth by plotting
log\(W_{\lambda}\) against log\(\tau(0)\):}

    \begin{Verbatim}[commandchars=\\\{\}]
{\color{incolor}In [{\color{incolor}21}]:} \PY{n}{Ts}\PY{o}{=}\PY{l+m+mi}{5700}
         \PY{n}{Tl}\PY{o}{=}\PY{l+m+mi}{4200}
         \PY{n}{tau0} \PY{o}{=} \PY{l+m+mi}{10}\PY{o}{*}\PY{o}{*}\PY{p}{(}\PY{n}{np}\PY{o}{.}\PY{n}{arange}\PY{p}{(}\PY{l+m+mi}{0}\PY{p}{,}\PY{l+m+mi}{61}\PY{p}{,}\PY{l+m+mi}{1}\PY{p}{)}\PY{o}{/}\PY{l+m+mi}{10} \PY{o}{\PYZhy{}} \PY{l+m+mi}{2}\PY{p}{)}
         \PY{n}{eqw} \PY{o}{=} \PY{n}{np}\PY{o}{.}\PY{n}{zeros\PYZus{}like}\PY{p}{(}\PY{n}{tau0}\PY{p}{)}
         \PY{n}{a}\PY{o}{=}\PY{l+m+mf}{0.1}
         \PY{k}{for} \PY{n}{i}\PY{p}{,}\PY{n}{t} \PY{o+ow}{in} \PY{n+nb}{enumerate}\PY{p}{(}\PY{n}{tau0}\PY{p}{)}\PY{p}{:}
             \PY{n}{integ} \PY{o}{=} \PY{n}{profile}\PY{p}{(}\PY{n}{a}\PY{p}{,}\PY{n}{t}\PY{p}{,}\PY{n}{u}\PY{p}{)}
             \PY{n}{reldepth} \PY{o}{=} \PY{p}{(}\PY{n}{integ}\PY{p}{[}\PY{l+m+mi}{0}\PY{p}{]} \PY{o}{\PYZhy{}} \PY{n}{integ}\PY{p}{)}\PY{o}{/}\PY{n}{integ}\PY{p}{[}\PY{l+m+mi}{0}\PY{p}{]}
             \PY{n}{eqw}\PY{p}{[}\PY{n}{i}\PY{p}{]} \PY{o}{=} \PY{n}{np}\PY{o}{.}\PY{n}{sum}\PY{p}{(}\PY{n}{reldepth}\PY{p}{)}\PY{o}{*}\PY{l+m+mf}{0.4}
         
         \PY{n}{plt}\PY{o}{.}\PY{n}{loglog}\PY{p}{(}\PY{n}{tau0}\PY{p}{,}\PY{n}{eqw}\PY{p}{)}
         \PY{n}{plt}\PY{o}{.}\PY{n}{xlabel}\PY{p}{(}\PY{l+s}{\PYZsq{}}\PY{l+s}{tau0}\PY{l+s}{\PYZsq{}}\PY{p}{)}
         \PY{n}{plt}\PY{o}{.}\PY{n}{ylabel}\PY{p}{(}\PY{l+s}{\PYZsq{}}\PY{l+s}{equivalent width}\PY{l+s}{\PYZsq{}}\PY{p}{)}
         \PY{n}{plt}\PY{o}{.}\PY{n}{show}\PY{p}{(}\PY{p}{)}
\end{Verbatim}

    \begin{center}
    \adjustimage{max size={0.9\linewidth}{0.9\paperheight}}{Lab3_files/Lab3_48_0.png}
    \end{center}
    { \hspace*{\fill} \\}
    
    \textbf{Explain what happens in the three different parts.} In the low
opacity limit, the equivalent width grows as the line depth grows. As
\(\tau \rightarrow 1\) the line profile begins to saturate and the
growth slows. At high opacity the wings grow due to the opacity becoming
high enough for high density processes take over therefore increasing
the equivalent width.

\textbf{The first part has slope 1:1, the third part has slope 1:2 in
this log-log plot. Why?} The first part of the slope increses linearly
as the population of ground state systems increase (as in Lab2). The
third part has a different slope due to the fact that the wings are
being filled out and is no longer linearly dependent on the population
of the lower state.

\textbf{Which parameter controls the location of the onset of the third
part? Give a rough estimate of its value for solar iron lines through
comparison with Figure 14.} It is caused by \(a\) as it determines the
strength of the Lorentzian. It appears to be similar to our plots thus
the solar iron lines should have \(a \approx 0.1\).

\textbf{Final question: of which parameter should you raise the
numerical value in order to produce emission lines instead of absorption
lines? Change it accordingly and rerun your programs to produce emission
profiles and an emission-line curve of growth. Avoid plotting negative W
values logarithmically by:}

By changing the temperature of the layer \(T_{layer}\) to be greater
than the surface temperature we get emission lines.

    \begin{Verbatim}[commandchars=\\\{\}]
{\color{incolor}In [{\color{incolor}22}]:} \PY{n}{Ts}\PY{o}{=}\PY{l+m+mi}{5700} \PY{c}{\PYZsh{} solar surface temperature}
         \PY{n}{Tl}\PY{o}{=}\PY{l+m+mi}{10000} \PY{c}{\PYZsh{} solar T\PYZhy{}min temperature = `reversing layer\PYZsq{}}
         
         \PY{k}{def} \PY{n+nf}{profile2}\PY{p}{(}\PY{n}{a}\PY{p}{,}\PY{n}{tau0}\PY{p}{,}\PY{n}{u}\PY{p}{)}\PY{p}{:}
             \PY{l+s+sd}{\PYZdq{}\PYZdq{}\PYZdq{} }
         \PY{l+s+sd}{    return a Schuster\PYZhy{}Schwarzschild profile}
         \PY{l+s+sd}{    input: a = damping parameter}
         \PY{l+s+sd}{        tau0 = SS layer thickness at line center}
         \PY{l+s+sd}{        u = wavelength array in Doppler units}
         \PY{l+s+sd}{    output: int = intensity array}
         \PY{l+s+sd}{    \PYZdq{}\PYZdq{}\PYZdq{}}
             \PY{n}{integ}\PY{o}{=}\PY{n}{np}\PY{o}{.}\PY{n}{zeros\PYZus{}like}\PY{p}{(}\PY{n}{u}\PY{p}{)}
             \PY{n}{usize}\PY{o}{=}\PY{n+nb}{len}\PY{p}{(}\PY{n}{u}\PY{p}{)} \PY{c}{\PYZsh{} IDL SIZE returns array type and dimensions}
             \PY{k}{for} \PY{n}{i} \PY{o+ow}{in} \PY{n+nb}{range}\PY{p}{(}\PY{n}{usize}\PY{p}{)}\PY{p}{:}
                 \PY{n}{tau}\PY{o}{=}\PY{n}{tau0} \PY{o}{*} \PY{n}{V2}\PY{p}{(}\PY{n}{a}\PY{p}{,}\PY{n+nb}{abs}\PY{p}{(}\PY{n}{u}\PY{p}{[}\PY{n}{i}\PY{p}{]}\PY{p}{)}\PY{p}{)}
                 \PY{n}{integ}\PY{p}{[}\PY{n}{i}\PY{p}{]}\PY{o}{=}\PY{n}{planck}\PY{p}{(}\PY{n}{Ts}\PY{p}{,}\PY{n}{wav}\PY{p}{)} \PY{o}{*} \PY{n}{np}\PY{o}{.}\PY{n}{exp}\PY{p}{(}\PY{o}{\PYZhy{}}\PY{n}{tau}\PY{p}{)} \PY{o}{+} \PY{n}{planck}\PY{p}{(}\PY{n}{Tl}\PY{p}{,}\PY{n}{wav}\PY{p}{)}\PY{o}{*}\PY{p}{(}\PY{l+m+mf}{1.}\PY{o}{\PYZhy{}}\PY{n}{np}\PY{o}{.}\PY{n}{exp}\PY{p}{(}\PY{o}{\PYZhy{}}\PY{n}{tau}\PY{p}{)}\PY{p}{)}
             \PY{k}{return} \PY{n}{integ}
         
         
         \PY{n}{a}\PY{o}{=}\PY{l+m+mf}{0.1} \PY{c}{\PYZsh{} damping parameter}
         \PY{n}{wav}\PY{o}{=}\PY{l+m+mf}{10000e\PYZhy{}8} \PY{c}{\PYZsh{} wavelength in cm}
         \PY{n}{tau0}\PY{o}{=}\PY{l+m+mi}{1} \PY{c}{\PYZsh{} reversing layer thickness at line center}
         \PY{c}{\PYZsh{}u=np.arange(201)/10.\PYZhy{}10. \PYZsh{} u = \PYZhy{}10 to 10 in 0.1 steps}
         \PY{n}{u} \PY{o}{=} \PY{n}{np}\PY{o}{.}\PY{n}{linspace}\PY{p}{(}\PY{o}{\PYZhy{}}\PY{l+m+mi}{10}\PY{p}{,}\PY{l+m+mi}{10}\PY{p}{,}\PY{l+m+mi}{200}\PY{p}{)}
         \PY{n}{integ}\PY{o}{=}\PY{n}{np}\PY{o}{.}\PY{n}{zeros\PYZus{}like}\PY{p}{(}\PY{n}{u}\PY{p}{)} \PY{c}{\PYZsh{} declare array}
         \PY{k}{for} \PY{n}{i} \PY{o+ow}{in} \PY{n+nb}{range}\PY{p}{(}\PY{l+m+mi}{200}\PY{p}{)}\PY{p}{:}
             \PY{n}{tau}\PY{o}{=}\PY{n}{tau0} \PY{o}{*} \PY{n}{V2}\PY{p}{(}\PY{n}{a}\PY{p}{,}\PY{n+nb}{abs}\PY{p}{(}\PY{n}{u}\PY{p}{[}\PY{n}{i}\PY{p}{]}\PY{p}{)}\PY{p}{)}
             \PY{n}{integ}\PY{p}{[}\PY{n}{i}\PY{p}{]}\PY{o}{=}\PY{n}{planck}\PY{p}{(}\PY{n}{Ts}\PY{p}{,}\PY{n}{wav}\PY{p}{)} \PY{o}{*} \PY{n}{np}\PY{o}{.}\PY{n}{exp}\PY{p}{(}\PY{o}{\PYZhy{}}\PY{n}{tau}\PY{p}{)} \PY{o}{+} \PY{n}{planck}\PY{p}{(}\PY{n}{Tl}\PY{p}{,}\PY{n}{wav}\PY{p}{)}\PY{o}{*}\PY{p}{(}\PY{l+m+mf}{1.}\PY{o}{\PYZhy{}}\PY{n}{np}\PY{o}{.}\PY{n}{exp}\PY{p}{(}\PY{o}{\PYZhy{}}\PY{n}{tau}\PY{p}{)}\PY{p}{)}
         
         \PY{c}{\PYZsh{}plt.plot(u,integ)}
         
         \PY{n}{tau0}\PY{o}{=}\PY{p}{[}\PY{l+m+mf}{0.01}\PY{p}{,} \PY{l+m+mf}{0.05}\PY{p}{,} \PY{l+m+mf}{0.1}\PY{p}{,} \PY{l+m+mf}{0.5}\PY{p}{,} \PY{l+m+mi}{1}\PY{p}{,} \PY{l+m+mi}{5}\PY{p}{,} \PY{l+m+mi}{10}\PY{p}{,} \PY{l+m+mi}{50}\PY{p}{,} \PY{l+m+mi}{100}\PY{p}{]}
         \PY{k}{for} \PY{n}{t} \PY{o+ow}{in} \PY{n}{tau0}\PY{p}{:}
             \PY{k}{for} \PY{n}{i} \PY{o+ow}{in} \PY{n+nb}{range}\PY{p}{(}\PY{l+m+mi}{200}\PY{p}{)}\PY{p}{:}
                 \PY{n}{tau} \PY{o}{=} \PY{n}{t}\PY{o}{*}\PY{n}{V2}\PY{p}{(}\PY{n}{a}\PY{p}{,}\PY{n+nb}{abs}\PY{p}{(}\PY{n}{u}\PY{p}{[}\PY{n}{i}\PY{p}{]}\PY{p}{)}\PY{p}{)} 
                 \PY{n}{integ}\PY{p}{[}\PY{n}{i}\PY{p}{]}\PY{o}{=}\PY{n}{planck}\PY{p}{(}\PY{n}{Ts}\PY{p}{,}\PY{n}{wav}\PY{p}{)} \PY{o}{*} \PY{n}{np}\PY{o}{.}\PY{n}{exp}\PY{p}{(}\PY{o}{\PYZhy{}}\PY{n}{tau}\PY{p}{)} \PY{o}{+} \PY{n}{planck}\PY{p}{(}\PY{n}{Tl}\PY{p}{,}\PY{n}{wav}\PY{p}{)}\PY{o}{*}\PY{p}{(}\PY{l+m+mf}{1.}\PY{o}{\PYZhy{}}\PY{n}{np}\PY{o}{.}\PY{n}{exp}\PY{p}{(}\PY{o}{\PYZhy{}}\PY{n}{tau}\PY{p}{)}\PY{p}{)}
             \PY{n}{plt}\PY{o}{.}\PY{n}{plot}\PY{p}{(}\PY{n}{u}\PY{p}{,}\PY{n}{integ}\PY{p}{,} \PY{n}{label}\PY{o}{=}\PY{l+s}{\PYZsq{}}\PY{l+s}{\PYZdl{}}\PY{l+s+se}{\PYZbs{}\PYZbs{}}\PY{l+s}{tau = \PYZob{}\PYZcb{}\PYZdl{}}\PY{l+s}{\PYZsq{}}\PY{o}{.}\PY{n}{format}\PY{p}{(}\PY{n}{t}\PY{p}{)}\PY{p}{)}
         
         \PY{n}{plt}\PY{o}{.}\PY{n}{legend}\PY{p}{(}\PY{p}{)}
         \PY{n}{plt}\PY{o}{.}\PY{n}{show}\PY{p}{(}\PY{p}{)}
\end{Verbatim}

    \begin{center}
    \adjustimage{max size={0.9\linewidth}{0.9\paperheight}}{Lab3_files/Lab3_50_0.png}
    \end{center}
    { \hspace*{\fill} \\}
    
    \begin{Verbatim}[commandchars=\\\{\}]
{\color{incolor}In [{\color{incolor}23}]:} \PY{n}{plt}\PY{o}{.}\PY{n}{figure}\PY{p}{(}\PY{p}{)}
         \PY{n}{tau0} \PY{o}{=} \PY{l+m+mi}{10}\PY{o}{*}\PY{o}{*}\PY{p}{(}\PY{n}{np}\PY{o}{.}\PY{n}{arange}\PY{p}{(}\PY{l+m+mi}{0}\PY{p}{,}\PY{l+m+mi}{61}\PY{p}{,}\PY{l+m+mi}{1}\PY{p}{)}\PY{o}{/}\PY{l+m+mi}{10} \PY{o}{\PYZhy{}} \PY{l+m+mi}{2}\PY{p}{)}
         \PY{n}{eqw} \PY{o}{=} \PY{n}{np}\PY{o}{.}\PY{n}{zeros\PYZus{}like}\PY{p}{(}\PY{n}{tau0}\PY{p}{)}
         \PY{k}{for} \PY{n}{i}\PY{p}{,}\PY{n}{t} \PY{o+ow}{in} \PY{n+nb}{enumerate}\PY{p}{(}\PY{n}{tau0}\PY{p}{)}\PY{p}{:}
             \PY{n}{integ} \PY{o}{=} \PY{n}{profile2}\PY{p}{(}\PY{n}{a}\PY{p}{,}\PY{n}{t}\PY{p}{,}\PY{n}{u}\PY{p}{)}
             \PY{n}{reldepth} \PY{o}{=} \PY{p}{(}\PY{n}{integ}\PY{p}{[}\PY{l+m+mi}{0}\PY{p}{]} \PY{o}{\PYZhy{}} \PY{n}{integ}\PY{p}{)}\PY{o}{/}\PY{n}{integ}\PY{p}{[}\PY{l+m+mi}{0}\PY{p}{]}
             \PY{n}{eqw}\PY{p}{[}\PY{n}{i}\PY{p}{]} \PY{o}{=} \PY{n}{np}\PY{o}{.}\PY{n}{sum}\PY{p}{(}\PY{n}{reldepth}\PY{p}{)}\PY{o}{*}\PY{l+m+mf}{0.4}
         
         \PY{n}{plt}\PY{o}{.}\PY{n}{loglog}\PY{p}{(}\PY{n}{tau0}\PY{p}{,}\PY{n+nb}{abs}\PY{p}{(}\PY{n}{eqw}\PY{p}{)}\PY{p}{)}
         \PY{n}{plt}\PY{o}{.}\PY{n}{xlabel}\PY{p}{(}\PY{l+s}{\PYZsq{}}\PY{l+s}{tau0}\PY{l+s}{\PYZsq{}}\PY{p}{)}
         \PY{n}{plt}\PY{o}{.}\PY{n}{ylabel}\PY{p}{(}\PY{l+s}{\PYZsq{}}\PY{l+s}{equivalent width}\PY{l+s}{\PYZsq{}}\PY{p}{)}
         \PY{n}{plt}\PY{o}{.}\PY{n}{show}\PY{p}{(}\PY{p}{)}
\end{Verbatim}

    \begin{center}
    \adjustimage{max size={0.9\linewidth}{0.9\paperheight}}{Lab3_files/Lab3_51_0.png}
    \end{center}
    { \hspace*{\fill} \\}
    
    \begin{Verbatim}[commandchars=\\\{\}]
{\color{incolor}In [{\color{incolor} }]:} 
\end{Verbatim}


    % Add a bibliography block to the postdoc
    
    
    
    \end{document}
