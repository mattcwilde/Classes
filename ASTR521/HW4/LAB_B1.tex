
% Default to the notebook output style

    


% Inherit from the specified cell style.




    
\documentclass{article}

    
    
    \usepackage{graphicx} % Used to insert images
    \usepackage{adjustbox} % Used to constrain images to a maximum size 
    \usepackage{color} % Allow colors to be defined
    \usepackage{enumerate} % Needed for markdown enumerations to work
    \usepackage{geometry} % Used to adjust the document margins
    \usepackage{amsmath} % Equations
    \usepackage{amssymb} % Equations
    \usepackage{eurosym} % defines \euro
    \usepackage[mathletters]{ucs} % Extended unicode (utf-8) support
    \usepackage[utf8x]{inputenc} % Allow utf-8 characters in the tex document
    \usepackage{fancyvrb} % verbatim replacement that allows latex
    \usepackage{grffile} % extends the file name processing of package graphics 
                         % to support a larger range 
    % The hyperref package gives us a pdf with properly built
    % internal navigation ('pdf bookmarks' for the table of contents,
    % internal cross-reference links, web links for URLs, etc.)
    \usepackage{hyperref}
    \usepackage{longtable} % longtable support required by pandoc >1.10
    \usepackage{booktabs}  % table support for pandoc > 1.12.2
    

    
    
    \definecolor{orange}{cmyk}{0,0.4,0.8,0.2}
    \definecolor{darkorange}{rgb}{.71,0.21,0.01}
    \definecolor{darkgreen}{rgb}{.12,.54,.11}
    \definecolor{myteal}{rgb}{.26, .44, .56}
    \definecolor{gray}{gray}{0.45}
    \definecolor{lightgray}{gray}{.95}
    \definecolor{mediumgray}{gray}{.8}
    \definecolor{inputbackground}{rgb}{.95, .95, .85}
    \definecolor{outputbackground}{rgb}{.95, .95, .95}
    \definecolor{traceback}{rgb}{1, .95, .95}
    % ansi colors
    \definecolor{red}{rgb}{.6,0,0}
    \definecolor{green}{rgb}{0,.65,0}
    \definecolor{brown}{rgb}{0.6,0.6,0}
    \definecolor{blue}{rgb}{0,.145,.698}
    \definecolor{purple}{rgb}{.698,.145,.698}
    \definecolor{cyan}{rgb}{0,.698,.698}
    \definecolor{lightgray}{gray}{0.5}
    
    % bright ansi colors
    \definecolor{darkgray}{gray}{0.25}
    \definecolor{lightred}{rgb}{1.0,0.39,0.28}
    \definecolor{lightgreen}{rgb}{0.48,0.99,0.0}
    \definecolor{lightblue}{rgb}{0.53,0.81,0.92}
    \definecolor{lightpurple}{rgb}{0.87,0.63,0.87}
    \definecolor{lightcyan}{rgb}{0.5,1.0,0.83}
    
    % commands and environments needed by pandoc snippets
    % extracted from the output of `pandoc -s`
    \providecommand{\tightlist}{%
      \setlength{\itemsep}{0pt}\setlength{\parskip}{0pt}}
    \DefineVerbatimEnvironment{Highlighting}{Verbatim}{commandchars=\\\{\}}
    % Add ',fontsize=\small' for more characters per line
    \newenvironment{Shaded}{}{}
    \newcommand{\KeywordTok}[1]{\textcolor[rgb]{0.00,0.44,0.13}{\textbf{{#1}}}}
    \newcommand{\DataTypeTok}[1]{\textcolor[rgb]{0.56,0.13,0.00}{{#1}}}
    \newcommand{\DecValTok}[1]{\textcolor[rgb]{0.25,0.63,0.44}{{#1}}}
    \newcommand{\BaseNTok}[1]{\textcolor[rgb]{0.25,0.63,0.44}{{#1}}}
    \newcommand{\FloatTok}[1]{\textcolor[rgb]{0.25,0.63,0.44}{{#1}}}
    \newcommand{\CharTok}[1]{\textcolor[rgb]{0.25,0.44,0.63}{{#1}}}
    \newcommand{\StringTok}[1]{\textcolor[rgb]{0.25,0.44,0.63}{{#1}}}
    \newcommand{\CommentTok}[1]{\textcolor[rgb]{0.38,0.63,0.69}{\textit{{#1}}}}
    \newcommand{\OtherTok}[1]{\textcolor[rgb]{0.00,0.44,0.13}{{#1}}}
    \newcommand{\AlertTok}[1]{\textcolor[rgb]{1.00,0.00,0.00}{\textbf{{#1}}}}
    \newcommand{\FunctionTok}[1]{\textcolor[rgb]{0.02,0.16,0.49}{{#1}}}
    \newcommand{\RegionMarkerTok}[1]{{#1}}
    \newcommand{\ErrorTok}[1]{\textcolor[rgb]{1.00,0.00,0.00}{\textbf{{#1}}}}
    \newcommand{\NormalTok}[1]{{#1}}
    
    % Define a nice break command that doesn't care if a line doesn't already
    % exist.
    \def\br{\hspace*{\fill} \\* }
    % Math Jax compatability definitions
    \def\gt{>}
    \def\lt{<}
    % Document parameters
    \title{LAB\_B1}
    
    
    

    % Pygments definitions
    
\makeatletter
\def\PY@reset{\let\PY@it=\relax \let\PY@bf=\relax%
    \let\PY@ul=\relax \let\PY@tc=\relax%
    \let\PY@bc=\relax \let\PY@ff=\relax}
\def\PY@tok#1{\csname PY@tok@#1\endcsname}
\def\PY@toks#1+{\ifx\relax#1\empty\else%
    \PY@tok{#1}\expandafter\PY@toks\fi}
\def\PY@do#1{\PY@bc{\PY@tc{\PY@ul{%
    \PY@it{\PY@bf{\PY@ff{#1}}}}}}}
\def\PY#1#2{\PY@reset\PY@toks#1+\relax+\PY@do{#2}}

\expandafter\def\csname PY@tok@gd\endcsname{\def\PY@tc##1{\textcolor[rgb]{0.63,0.00,0.00}{##1}}}
\expandafter\def\csname PY@tok@gu\endcsname{\let\PY@bf=\textbf\def\PY@tc##1{\textcolor[rgb]{0.50,0.00,0.50}{##1}}}
\expandafter\def\csname PY@tok@gt\endcsname{\def\PY@tc##1{\textcolor[rgb]{0.00,0.27,0.87}{##1}}}
\expandafter\def\csname PY@tok@gs\endcsname{\let\PY@bf=\textbf}
\expandafter\def\csname PY@tok@gr\endcsname{\def\PY@tc##1{\textcolor[rgb]{1.00,0.00,0.00}{##1}}}
\expandafter\def\csname PY@tok@cm\endcsname{\let\PY@it=\textit\def\PY@tc##1{\textcolor[rgb]{0.25,0.50,0.50}{##1}}}
\expandafter\def\csname PY@tok@vg\endcsname{\def\PY@tc##1{\textcolor[rgb]{0.10,0.09,0.49}{##1}}}
\expandafter\def\csname PY@tok@m\endcsname{\def\PY@tc##1{\textcolor[rgb]{0.40,0.40,0.40}{##1}}}
\expandafter\def\csname PY@tok@mh\endcsname{\def\PY@tc##1{\textcolor[rgb]{0.40,0.40,0.40}{##1}}}
\expandafter\def\csname PY@tok@go\endcsname{\def\PY@tc##1{\textcolor[rgb]{0.53,0.53,0.53}{##1}}}
\expandafter\def\csname PY@tok@ge\endcsname{\let\PY@it=\textit}
\expandafter\def\csname PY@tok@vc\endcsname{\def\PY@tc##1{\textcolor[rgb]{0.10,0.09,0.49}{##1}}}
\expandafter\def\csname PY@tok@il\endcsname{\def\PY@tc##1{\textcolor[rgb]{0.40,0.40,0.40}{##1}}}
\expandafter\def\csname PY@tok@cs\endcsname{\let\PY@it=\textit\def\PY@tc##1{\textcolor[rgb]{0.25,0.50,0.50}{##1}}}
\expandafter\def\csname PY@tok@cp\endcsname{\def\PY@tc##1{\textcolor[rgb]{0.74,0.48,0.00}{##1}}}
\expandafter\def\csname PY@tok@gi\endcsname{\def\PY@tc##1{\textcolor[rgb]{0.00,0.63,0.00}{##1}}}
\expandafter\def\csname PY@tok@gh\endcsname{\let\PY@bf=\textbf\def\PY@tc##1{\textcolor[rgb]{0.00,0.00,0.50}{##1}}}
\expandafter\def\csname PY@tok@ni\endcsname{\let\PY@bf=\textbf\def\PY@tc##1{\textcolor[rgb]{0.60,0.60,0.60}{##1}}}
\expandafter\def\csname PY@tok@nl\endcsname{\def\PY@tc##1{\textcolor[rgb]{0.63,0.63,0.00}{##1}}}
\expandafter\def\csname PY@tok@nn\endcsname{\let\PY@bf=\textbf\def\PY@tc##1{\textcolor[rgb]{0.00,0.00,1.00}{##1}}}
\expandafter\def\csname PY@tok@no\endcsname{\def\PY@tc##1{\textcolor[rgb]{0.53,0.00,0.00}{##1}}}
\expandafter\def\csname PY@tok@na\endcsname{\def\PY@tc##1{\textcolor[rgb]{0.49,0.56,0.16}{##1}}}
\expandafter\def\csname PY@tok@nb\endcsname{\def\PY@tc##1{\textcolor[rgb]{0.00,0.50,0.00}{##1}}}
\expandafter\def\csname PY@tok@nc\endcsname{\let\PY@bf=\textbf\def\PY@tc##1{\textcolor[rgb]{0.00,0.00,1.00}{##1}}}
\expandafter\def\csname PY@tok@nd\endcsname{\def\PY@tc##1{\textcolor[rgb]{0.67,0.13,1.00}{##1}}}
\expandafter\def\csname PY@tok@ne\endcsname{\let\PY@bf=\textbf\def\PY@tc##1{\textcolor[rgb]{0.82,0.25,0.23}{##1}}}
\expandafter\def\csname PY@tok@nf\endcsname{\def\PY@tc##1{\textcolor[rgb]{0.00,0.00,1.00}{##1}}}
\expandafter\def\csname PY@tok@si\endcsname{\let\PY@bf=\textbf\def\PY@tc##1{\textcolor[rgb]{0.73,0.40,0.53}{##1}}}
\expandafter\def\csname PY@tok@s2\endcsname{\def\PY@tc##1{\textcolor[rgb]{0.73,0.13,0.13}{##1}}}
\expandafter\def\csname PY@tok@vi\endcsname{\def\PY@tc##1{\textcolor[rgb]{0.10,0.09,0.49}{##1}}}
\expandafter\def\csname PY@tok@nt\endcsname{\let\PY@bf=\textbf\def\PY@tc##1{\textcolor[rgb]{0.00,0.50,0.00}{##1}}}
\expandafter\def\csname PY@tok@nv\endcsname{\def\PY@tc##1{\textcolor[rgb]{0.10,0.09,0.49}{##1}}}
\expandafter\def\csname PY@tok@s1\endcsname{\def\PY@tc##1{\textcolor[rgb]{0.73,0.13,0.13}{##1}}}
\expandafter\def\csname PY@tok@kd\endcsname{\let\PY@bf=\textbf\def\PY@tc##1{\textcolor[rgb]{0.00,0.50,0.00}{##1}}}
\expandafter\def\csname PY@tok@sh\endcsname{\def\PY@tc##1{\textcolor[rgb]{0.73,0.13,0.13}{##1}}}
\expandafter\def\csname PY@tok@sc\endcsname{\def\PY@tc##1{\textcolor[rgb]{0.73,0.13,0.13}{##1}}}
\expandafter\def\csname PY@tok@sx\endcsname{\def\PY@tc##1{\textcolor[rgb]{0.00,0.50,0.00}{##1}}}
\expandafter\def\csname PY@tok@bp\endcsname{\def\PY@tc##1{\textcolor[rgb]{0.00,0.50,0.00}{##1}}}
\expandafter\def\csname PY@tok@c1\endcsname{\let\PY@it=\textit\def\PY@tc##1{\textcolor[rgb]{0.25,0.50,0.50}{##1}}}
\expandafter\def\csname PY@tok@kc\endcsname{\let\PY@bf=\textbf\def\PY@tc##1{\textcolor[rgb]{0.00,0.50,0.00}{##1}}}
\expandafter\def\csname PY@tok@c\endcsname{\let\PY@it=\textit\def\PY@tc##1{\textcolor[rgb]{0.25,0.50,0.50}{##1}}}
\expandafter\def\csname PY@tok@mf\endcsname{\def\PY@tc##1{\textcolor[rgb]{0.40,0.40,0.40}{##1}}}
\expandafter\def\csname PY@tok@err\endcsname{\def\PY@bc##1{\setlength{\fboxsep}{0pt}\fcolorbox[rgb]{1.00,0.00,0.00}{1,1,1}{\strut ##1}}}
\expandafter\def\csname PY@tok@mb\endcsname{\def\PY@tc##1{\textcolor[rgb]{0.40,0.40,0.40}{##1}}}
\expandafter\def\csname PY@tok@ss\endcsname{\def\PY@tc##1{\textcolor[rgb]{0.10,0.09,0.49}{##1}}}
\expandafter\def\csname PY@tok@sr\endcsname{\def\PY@tc##1{\textcolor[rgb]{0.73,0.40,0.53}{##1}}}
\expandafter\def\csname PY@tok@mo\endcsname{\def\PY@tc##1{\textcolor[rgb]{0.40,0.40,0.40}{##1}}}
\expandafter\def\csname PY@tok@kn\endcsname{\let\PY@bf=\textbf\def\PY@tc##1{\textcolor[rgb]{0.00,0.50,0.00}{##1}}}
\expandafter\def\csname PY@tok@mi\endcsname{\def\PY@tc##1{\textcolor[rgb]{0.40,0.40,0.40}{##1}}}
\expandafter\def\csname PY@tok@gp\endcsname{\let\PY@bf=\textbf\def\PY@tc##1{\textcolor[rgb]{0.00,0.00,0.50}{##1}}}
\expandafter\def\csname PY@tok@o\endcsname{\def\PY@tc##1{\textcolor[rgb]{0.40,0.40,0.40}{##1}}}
\expandafter\def\csname PY@tok@kr\endcsname{\let\PY@bf=\textbf\def\PY@tc##1{\textcolor[rgb]{0.00,0.50,0.00}{##1}}}
\expandafter\def\csname PY@tok@s\endcsname{\def\PY@tc##1{\textcolor[rgb]{0.73,0.13,0.13}{##1}}}
\expandafter\def\csname PY@tok@kp\endcsname{\def\PY@tc##1{\textcolor[rgb]{0.00,0.50,0.00}{##1}}}
\expandafter\def\csname PY@tok@w\endcsname{\def\PY@tc##1{\textcolor[rgb]{0.73,0.73,0.73}{##1}}}
\expandafter\def\csname PY@tok@kt\endcsname{\def\PY@tc##1{\textcolor[rgb]{0.69,0.00,0.25}{##1}}}
\expandafter\def\csname PY@tok@ow\endcsname{\let\PY@bf=\textbf\def\PY@tc##1{\textcolor[rgb]{0.67,0.13,1.00}{##1}}}
\expandafter\def\csname PY@tok@sb\endcsname{\def\PY@tc##1{\textcolor[rgb]{0.73,0.13,0.13}{##1}}}
\expandafter\def\csname PY@tok@k\endcsname{\let\PY@bf=\textbf\def\PY@tc##1{\textcolor[rgb]{0.00,0.50,0.00}{##1}}}
\expandafter\def\csname PY@tok@se\endcsname{\let\PY@bf=\textbf\def\PY@tc##1{\textcolor[rgb]{0.73,0.40,0.13}{##1}}}
\expandafter\def\csname PY@tok@sd\endcsname{\let\PY@it=\textit\def\PY@tc##1{\textcolor[rgb]{0.73,0.13,0.13}{##1}}}

\def\PYZbs{\char`\\}
\def\PYZus{\char`\_}
\def\PYZob{\char`\{}
\def\PYZcb{\char`\}}
\def\PYZca{\char`\^}
\def\PYZam{\char`\&}
\def\PYZlt{\char`\<}
\def\PYZgt{\char`\>}
\def\PYZsh{\char`\#}
\def\PYZpc{\char`\%}
\def\PYZdl{\char`\$}
\def\PYZhy{\char`\-}
\def\PYZsq{\char`\'}
\def\PYZdq{\char`\"}
\def\PYZti{\char`\~}
% for compatibility with earlier versions
\def\PYZat{@}
\def\PYZlb{[}
\def\PYZrb{]}
\makeatother


    % Exact colors from NB
    \definecolor{incolor}{rgb}{0.0, 0.0, 0.5}
    \definecolor{outcolor}{rgb}{0.545, 0.0, 0.0}



    
    % Prevent overflowing lines due to hard-to-break entities
    \sloppy 
    % Setup hyperref package
    \hypersetup{
      breaklinks=true,  % so long urls are correctly broken across lines
      colorlinks=true,
      urlcolor=blue,
      linkcolor=darkorange,
      citecolor=darkgreen,
      }
    % Slightly bigger margins than the latex defaults
    
    \geometry{verbose,tmargin=1in,bmargin=1in,lmargin=1in,rmargin=1in}
    
    

    \begin{document}
    
    
    \maketitle
    
    

    
    \section*{1 Stratification of the solar
atmosphere}\label{stratification-of-the-solar-atmosphere}

\subsection*{1.1 FALC temperature
stratification}\label{falc-temperature-stratification}

\textbf{Figure 2 shows the FALC temperature stratification. It was made
with IDL code similar to the following. Write similar code and make it
work.}

    \begin{Verbatim}[commandchars=\\\{\}]
{\color{incolor}In [{\color{incolor}1}]:} \PY{o}{\PYZpc{}}\PY{k}{matplotlib} inline
        \PY{k+kn}{import} \PY{n+nn}{numpy} \PY{k+kn}{as} \PY{n+nn}{np}
        \PY{k+kn}{import} \PY{n+nn}{matplotlib.pyplot} \PY{k+kn}{as} \PY{n+nn}{plt}
        \PY{k+kn}{import} \PY{n+nn}{matplotlib} \PY{k+kn}{as} \PY{n+nn}{mpl}
        \PY{k+kn}{import} \PY{n+nn}{astropy.constants} \PY{k+kn}{as} \PY{n+nn}{const}
        
        \PY{c}{\PYZsh{}use the pretty LaTeX fonts}
        \PY{n}{mpl}\PY{o}{.}\PY{n}{rcParams}\PY{o}{.}\PY{n}{update}\PY{p}{(}\PY{p}{\PYZob{}}\PY{l+s}{\PYZsq{}}\PY{l+s}{text.usetex}\PY{l+s}{\PYZsq{}}\PY{p}{:} \PY{n+nb+bp}{True}\PY{p}{\PYZcb{}}\PY{p}{)}
        \PY{n}{plt}\PY{o}{.}\PY{n}{rc}\PY{p}{(}\PY{l+s}{\PYZsq{}}\PY{l+s}{font}\PY{l+s}{\PYZsq{}}\PY{p}{,} \PY{n}{family}\PY{o}{=}\PY{l+s}{\PYZsq{}}\PY{l+s}{serif}\PY{l+s}{\PYZsq{}}\PY{p}{,} \PY{n}{size}\PY{o}{=}\PY{l+m+mi}{10}\PY{p}{)}
        \PY{n}{mpl}\PY{o}{.}\PY{n}{rc}\PY{p}{(}\PY{l+s}{\PYZsq{}}\PY{l+s}{axes.formatter}\PY{l+s}{\PYZsq{}}\PY{p}{,} \PY{n}{useoffset}\PY{o}{=}\PY{n+nb+bp}{False}\PY{p}{)}
        
        \PY{c}{\PYZsh{} plt.style.use(\PYZsq{}ggplot\PYZsq{})}
\end{Verbatim}

    \begin{Verbatim}[commandchars=\\\{\}]
{\color{incolor}In [{\color{incolor}2}]:} \PY{n}{k} \PY{o}{=} \PY{n}{const}\PY{o}{.}\PY{n}{k\PYZus{}B}\PY{o}{.}\PY{n}{cgs}\PY{o}{.}\PY{n}{value}
\end{Verbatim}

    \begin{Verbatim}[commandchars=\\\{\}]
{\color{incolor}In [{\color{incolor}3}]:} \PY{o}{!}head falc.dat
\end{Verbatim}

    \begin{Verbatim}[commandchars=\\\{\}]
FALC solar model atmosphere of Fontenla, Avrett \& Loeser 1993ApJ{\ldots}406..319F; 82 heights top-to-bottom
 height   tau\_500    colmass    temp    v\_turb n\_Htotal   n\_proton   n\_electron pressure  p\_gas/p  density
 [km]     dimless    [g/cm\^{}2]   [K]     [km/s] [cm\^{}-3]    [cm\^{}-3]    [cm\^{}-3]    [dyn/cm2] ratio    [g/cm\^{}3]
 
 2218.20  0.000E+00  6.777E-06  100000  11.73  5.575E+09  5.575E+09  6.665E+09  1.857E-01  0.952  1.306E-14
 2216.50  7.696E-10  6.779E-06   95600  11.65  5.838E+09  5.837E+09  6.947E+09  1.857E-01  0.950  1.368E-14
 2214.89  1.531E-09  6.781E-06   90816  11.56  6.151E+09  6.150E+09  7.284E+09  1.858E-01  0.948  1.441E-14
 2212.77  2.597E-09  6.785E-06   83891  11.42  6.668E+09  6.667E+09  7.834E+09  1.859E-01  0.945  1.562E-14
 2210.64  3.754E-09  6.788E-06   75934  11.25  7.381E+09  7.378E+09  8.576E+09  1.860E-01  0.941  1.729E-14
 2209.57  4.384E-09  6.790E-06   71336  11.14  7.864E+09  7.858E+09  9.076E+09  1.860E-01  0.938  1.843E-14
    \end{Verbatim}

    \begin{Verbatim}[commandchars=\\\{\}]
{\color{incolor}In [{\color{incolor}4}]:} \PY{c}{\PYZsh{} read in the data}
        \PY{n}{h}\PY{p}{,} \PY{n}{tau5}\PY{p}{,} \PY{n}{colm}\PY{p}{,} \PY{n}{temp}\PY{p}{,} \PY{n}{vturb}\PY{p}{,} \PY{n}{nhyd}\PY{p}{,} \PY{n}{nprot}\PY{p}{,} \PY{n}{nel}\PY{p}{,} \PY{n}{ptot}\PY{p}{,} \PY{n}{pgasptot}\PY{p}{,} \PY{n}{dens} \PY{o}{=} \PY{n}{np}\PY{o}{.}\PY{n}{loadtxt}\PY{p}{(}\PY{l+s}{\PYZsq{}}\PY{l+s}{falc.dat}\PY{l+s}{\PYZsq{}}\PY{p}{,}\PY{n}{skiprows}\PY{o}{=}\PY{l+m+mi}{4}\PY{p}{,} \PY{n}{unpack}\PY{o}{=}\PY{n+nb+bp}{True}\PY{p}{)}
        
        \PY{n}{hsun} \PY{o}{=} \PY{n}{h}
\end{Verbatim}

    \begin{Verbatim}[commandchars=\\\{\}]
{\color{incolor}In [{\color{incolor}5}]:} \PY{n}{fig}\PY{p}{,} \PY{n}{ax} \PY{o}{=} \PY{n}{plt}\PY{o}{.}\PY{n}{subplots}\PY{p}{(}\PY{p}{)}
        \PY{n}{ax}\PY{o}{.}\PY{n}{plot}\PY{p}{(}\PY{n}{h}\PY{p}{,}\PY{n}{temp}\PY{p}{)}
        \PY{n}{ax}\PY{o}{.}\PY{n}{set\PYZus{}ylim}\PY{p}{(}\PY{l+m+mi}{3000}\PY{p}{,}\PY{l+m+mi}{10000}\PY{p}{)}
        \PY{n}{ax}\PY{o}{.}\PY{n}{set\PYZus{}xlabel}\PY{p}{(}\PY{l+s}{\PYZsq{}}\PY{l+s}{height [km]}\PY{l+s}{\PYZsq{}}\PY{p}{)}
        \PY{n}{ax}\PY{o}{.}\PY{n}{set\PYZus{}ylabel}\PY{p}{(}\PY{l+s}{\PYZsq{}}\PY{l+s}{temperature [K]}\PY{l+s}{\PYZsq{}}\PY{p}{)}
        \PY{n}{ax}\PY{o}{.}\PY{n}{set\PYZus{}title}\PY{p}{(}\PY{l+s}{\PYZsq{}}\PY{l+s}{FALC temperature stratification}\PY{l+s}{\PYZsq{}}\PY{p}{)}
        \PY{n}{plt}\PY{o}{.}\PY{n}{show}\PY{p}{(}\PY{p}{)}
\end{Verbatim}

    \begin{center}
    \adjustimage{max size={0.9\linewidth}{0.9\paperheight}}{LAB_B1_files/LAB_B1_5_0.png}
    \end{center}
    { \hspace*{\fill} \\}
    
    \subsection*{1.2 FALC density
stratification}\label{falc-density-stratification}

\textbf{Plot the total pressure \(p_{total}\) against the column mass
\(m\), both linearly and logarithmically. You will find that they scale
linearly. Exlpain what assumption has caused \(p_{total} = c m\) and
determine the value of solar surface gravity \(g = c\) that went into
the FALC-producing code.}

    \begin{Verbatim}[commandchars=\\\{\}]
{\color{incolor}In [{\color{incolor}6}]:} \PY{n}{fig}\PY{p}{,} \PY{n}{ax} \PY{o}{=} \PY{n}{plt}\PY{o}{.}\PY{n}{subplots}\PY{p}{(}\PY{n}{ncols}\PY{o}{=}\PY{l+m+mi}{2}\PY{p}{)}
        
        \PY{n}{ax}\PY{p}{[}\PY{l+m+mi}{0}\PY{p}{]}\PY{o}{.}\PY{n}{plot}\PY{p}{(}\PY{n}{colm}\PY{p}{,}\PY{n}{ptot}\PY{p}{)}
        \PY{n}{ax}\PY{p}{[}\PY{l+m+mi}{0}\PY{p}{]}\PY{o}{.}\PY{n}{set\PYZus{}xlabel}\PY{p}{(}\PY{l+s}{\PYZsq{}}\PY{l+s}{column mass [g cm\PYZdl{}\PYZca{}\PYZob{}\PYZhy{}2\PYZcb{}\PYZdl{}]}\PY{l+s}{\PYZsq{}}\PY{p}{)}
        \PY{n}{ax}\PY{p}{[}\PY{l+m+mi}{0}\PY{p}{]}\PY{o}{.}\PY{n}{set\PYZus{}ylabel}\PY{p}{(}\PY{l+s}{\PYZsq{}}\PY{l+s}{pressure [dyne cm\PYZdl{}\PYZca{}\PYZob{}\PYZhy{}2\PYZcb{}\PYZdl{}]}\PY{l+s}{\PYZsq{}}\PY{p}{)}
        
        \PY{n}{ax}\PY{p}{[}\PY{l+m+mi}{1}\PY{p}{]}\PY{o}{.}\PY{n}{loglog}\PY{p}{(}\PY{n}{colm}\PY{p}{,}\PY{n}{ptot}\PY{p}{)}
        \PY{n}{ax}\PY{p}{[}\PY{l+m+mi}{1}\PY{p}{]}\PY{o}{.}\PY{n}{set\PYZus{}xlabel}\PY{p}{(}\PY{l+s}{\PYZsq{}}\PY{l+s}{column mass [g cm\PYZdl{}\PYZca{}\PYZob{}\PYZhy{}2\PYZcb{}\PYZdl{}]}\PY{l+s}{\PYZsq{}}\PY{p}{)}
        
        \PY{n}{plt}\PY{o}{.}\PY{n}{show}\PY{p}{(}\PY{p}{)}
\end{Verbatim}

    \begin{center}
    \adjustimage{max size={0.9\linewidth}{0.9\paperheight}}{LAB_B1_files/LAB_B1_7_0.png}
    \end{center}
    { \hspace*{\fill} \\}
    
    We assumed hydrostatic equilibrium.

\[ \frac{P}{h} = \rho g  = \frac{m g}{V} = \frac{m_{col}g}{h} \rightarrow  P = gm \]

And \(g = 27396.5 \rightarrow log g = 4.4\)

    \begin{Verbatim}[commandchars=\\\{\}]
{\color{incolor}In [{\color{incolor}7}]:} \PY{n}{g} \PY{o}{=} \PY{p}{(}\PY{n}{ptot}\PY{p}{[}\PY{o}{\PYZhy{}}\PY{l+m+mi}{1}\PY{p}{]} \PY{o}{\PYZhy{}} \PY{n}{ptot}\PY{p}{[}\PY{l+m+mi}{0}\PY{p}{]}\PY{p}{)}\PY{o}{/}\PY{p}{(}\PY{n}{colm}\PY{p}{[}\PY{o}{\PYZhy{}}\PY{l+m+mi}{1}\PY{p}{]} \PY{o}{\PYZhy{}} \PY{n}{colm}\PY{p}{[}\PY{l+m+mi}{0}\PY{p}{]}\PY{p}{)} 
        \PY{k}{print} \PY{l+s}{\PYZdq{}}\PY{l+s}{solar g, logg = }\PY{l+s}{\PYZdq{}}\PY{p}{,}\PY{n}{g}\PY{p}{,} \PY{n}{np}\PY{o}{.}\PY{n}{log10}\PY{p}{(}\PY{n}{g}\PY{p}{)}
\end{Verbatim}

    \begin{Verbatim}[commandchars=\\\{\}]
solar g, logg =  27396.5215001 4.43769542453
    \end{Verbatim}

    \textbf{Fontenla et al (1993) also assumed complete mixing, i.e., the
same element mix at all heights. Check this by plotting the ratio of the
hydrogen mass density to the total mass density against height. Then add
helium to hydrogen using their abundance and mass ratios
(\(N_{He}/N_{H} = 0.1, m_{He} = 3.97 m_{H}\)), estimate the fraction of
the total mass density made up by the remaining elements in the model
mix (``the metals'').}

    \begin{Verbatim}[commandchars=\\\{\}]
{\color{incolor}In [{\color{incolor}8}]:} \PY{n}{mh} \PY{o}{=} \PY{l+m+mf}{1.67352e\PYZhy{}24} \PY{c}{\PYZsh{} mass of hydroen [g]}
        \PY{n}{H\PYZus{}ratio} \PY{o}{=} \PY{n}{nhyd}\PY{o}{*}\PY{n}{mh} \PY{o}{/} \PY{n}{dens}
        \PY{n}{fig}\PY{p}{,} \PY{n}{ax} \PY{o}{=} \PY{n}{plt}\PY{o}{.}\PY{n}{subplots}\PY{p}{(}\PY{p}{)}
        \PY{n}{ax}\PY{o}{.}\PY{n}{plot}\PY{p}{(}\PY{n}{h}\PY{p}{,}\PY{n}{H\PYZus{}ratio}\PY{p}{,} \PY{n}{label}\PY{o}{=}\PY{l+s}{\PYZsq{}}\PY{l+s}{H}\PY{l+s}{\PYZsq{}}\PY{p}{)}
        \PY{n}{ax}\PY{o}{.}\PY{n}{set\PYZus{}ylabel}\PY{p}{(}\PY{l+s}{\PYZsq{}}\PY{l+s}{density fraction of H}\PY{l+s}{\PYZsq{}}\PY{p}{)}
        \PY{n}{ax}\PY{o}{.}\PY{n}{set\PYZus{}xlabel}\PY{p}{(}\PY{l+s}{\PYZsq{}}\PY{l+s}{height [km]}\PY{l+s}{\PYZsq{}}\PY{p}{)}
        
        \PY{n}{plt}\PY{o}{.}\PY{n}{show}\PY{p}{(}\PY{p}{)}
\end{Verbatim}

    \begin{center}
    \adjustimage{max size={0.9\linewidth}{0.9\paperheight}}{LAB_B1_files/LAB_B1_11_0.png}
    \end{center}
    { \hspace*{\fill} \\}
    
    \begin{Verbatim}[commandchars=\\\{\}]
{\color{incolor}In [{\color{incolor}9}]:} \PY{n}{fig}\PY{p}{,} \PY{n}{ax} \PY{o}{=} \PY{n}{plt}\PY{o}{.}\PY{n}{subplots}\PY{p}{(}\PY{p}{)}
        \PY{n}{HHe\PYZus{}ratio} \PY{o}{=} \PY{p}{(}\PY{n}{nhyd}\PY{o}{*}\PY{n}{mh} \PY{o}{+} \PY{l+m+mf}{0.1}\PY{o}{*}\PY{n}{nhyd}\PY{o}{*}\PY{l+m+mf}{3.97}\PY{o}{*}\PY{n}{mh}\PY{p}{)} \PY{o}{/} \PY{n}{dens}
        \PY{n}{ax}\PY{o}{.}\PY{n}{plot}\PY{p}{(}\PY{n}{h}\PY{p}{,} \PY{n}{HHe\PYZus{}ratio}\PY{p}{)}
        \PY{n}{ax}\PY{o}{.}\PY{n}{set\PYZus{}ylabel}\PY{p}{(}\PY{l+s}{\PYZsq{}}\PY{l+s}{density fraction of H + He}\PY{l+s}{\PYZsq{}}\PY{p}{)}
        \PY{n}{ax}\PY{o}{.}\PY{n}{set\PYZus{}xlabel}\PY{p}{(}\PY{l+s}{\PYZsq{}}\PY{l+s}{height [km]}\PY{l+s}{\PYZsq{}}\PY{p}{)}
        \PY{n}{plt}\PY{o}{.}\PY{n}{show}\PY{p}{(}\PY{p}{)}
\end{Verbatim}

    \begin{center}
    \adjustimage{max size={0.9\linewidth}{0.9\paperheight}}{LAB_B1_files/LAB_B1_12_0.png}
    \end{center}
    { \hspace*{\fill} \\}
    
    \begin{Verbatim}[commandchars=\\\{\}]
{\color{incolor}In [{\color{incolor}10}]:} \PY{n}{fig}\PY{p}{,} \PY{n}{ax} \PY{o}{=} \PY{n}{plt}\PY{o}{.}\PY{n}{subplots}\PY{p}{(}\PY{p}{)}
         \PY{n}{metal\PYZus{}ratio} \PY{o}{=} \PY{l+m+mi}{1} \PY{o}{\PYZhy{}} \PY{p}{(}\PY{n}{nhyd}\PY{o}{*}\PY{n}{mh} \PY{o}{+} \PY{l+m+mf}{0.1}\PY{o}{*}\PY{n}{nhyd}\PY{o}{*}\PY{l+m+mf}{3.97}\PY{o}{*}\PY{n}{mh}\PY{p}{)}\PY{o}{/}\PY{n}{dens}
         \PY{n}{ax}\PY{o}{.}\PY{n}{plot}\PY{p}{(}\PY{n}{h}\PY{p}{,} \PY{n}{metal\PYZus{}ratio}\PY{p}{)}
         \PY{n}{ax}\PY{o}{.}\PY{n}{set\PYZus{}ylabel}\PY{p}{(}\PY{l+s}{\PYZsq{}}\PY{l+s}{density fraction of metals}\PY{l+s}{\PYZsq{}}\PY{p}{)}
         \PY{n}{ax}\PY{o}{.}\PY{n}{set\PYZus{}xlabel}\PY{p}{(}\PY{l+s}{\PYZsq{}}\PY{l+s}{height [km]}\PY{l+s}{\PYZsq{}}\PY{p}{)}
         \PY{n}{plt}\PY{o}{.}\PY{n}{show}\PY{p}{(}\PY{p}{)}
\end{Verbatim}

    \begin{center}
    \adjustimage{max size={0.9\linewidth}{0.9\paperheight}}{LAB_B1_files/LAB_B1_13_0.png}
    \end{center}
    { \hspace*{\fill} \\}
    
    \begin{Verbatim}[commandchars=\\\{\}]
{\color{incolor}In [{\color{incolor}11}]:} \PY{k}{print}\PY{p}{(}\PY{l+s}{r\PYZdq{}}\PY{l+s}{average metal fraction is \PYZti{} \PYZob{}:.2\PYZcb{}}\PY{l+s}{\PYZdq{}}\PY{o}{.}\PY{n}{format}\PY{p}{(}\PY{p}{(}\PY{l+m+mi}{1} \PY{o}{\PYZhy{}} \PY{n}{HHe\PYZus{}ratio}\PY{p}{)}\PY{o}{.}\PY{n}{mean}\PY{p}{(}\PY{p}{)}\PY{p}{)}\PY{p}{)}
\end{Verbatim}

    \begin{Verbatim}[commandchars=\\\{\}]
average metal fraction is \textasciitilde{} 0.0022
    \end{Verbatim}

    \textbf{Plot the column mass against height. The curve becomes nearly
straight when you make the \(y\)-axis logarithmic. Why is that? Why
isn't it exactly straight?}

It is neary straight since via hydrostaic equilibrium
\(log(m_{col}) = -h \mu g/kT\) but it depends on \(\mu\) which changes
slightly at different depths in the atmosphere as seen below.

    \begin{Verbatim}[commandchars=\\\{\}]
{\color{incolor}In [{\color{incolor}12}]:} \PY{n}{mu} \PY{o}{=} \PY{n}{dens}\PY{o}{/}\PY{p}{(}\PY{p}{(}\PY{n}{nhyd}\PY{o}{+}\PY{l+m+mf}{0.1}\PY{o}{*}\PY{n}{nhyd}\PY{o}{+}\PY{n}{nprot}\PY{p}{)}\PY{o}{*}\PY{n}{mh}\PY{p}{)}
         \PY{n}{plt}\PY{o}{.}\PY{n}{plot}\PY{p}{(}\PY{n}{h}\PY{p}{,}\PY{n}{mu}\PY{p}{)}
         \PY{n}{plt}\PY{o}{.}\PY{n}{xlabel}\PY{p}{(}\PY{l+s}{\PYZsq{}}\PY{l+s}{height [km]}\PY{l+s}{\PYZsq{}}\PY{p}{)}
         \PY{n}{plt}\PY{o}{.}\PY{n}{ylabel}\PY{p}{(}\PY{l+s}{r\PYZsq{}}\PY{l+s}{\PYZdl{}}\PY{l+s}{\PYZbs{}}\PY{l+s}{mu\PYZdl{}}\PY{l+s}{\PYZsq{}}\PY{p}{)}
\end{Verbatim}

            \begin{Verbatim}[commandchars=\\\{\}]
{\color{outcolor}Out[{\color{outcolor}12}]:} <matplotlib.text.Text at 0x106a74450>
\end{Verbatim}
        
    \begin{center}
    \adjustimage{max size={0.9\linewidth}{0.9\paperheight}}{LAB_B1_files/LAB_B1_16_1.png}
    \end{center}
    { \hspace*{\fill} \\}
    
    \begin{Verbatim}[commandchars=\\\{\}]
{\color{incolor}In [{\color{incolor}13}]:} \PY{n}{fig}\PY{p}{,} \PY{n}{ax} \PY{o}{=} \PY{n}{plt}\PY{o}{.}\PY{n}{subplots}\PY{p}{(}\PY{n}{ncols}\PY{o}{=}\PY{l+m+mi}{2}\PY{p}{)}
         \PY{n}{ax}\PY{p}{[}\PY{l+m+mi}{0}\PY{p}{]}\PY{o}{.}\PY{n}{plot}\PY{p}{(}\PY{n}{h}\PY{p}{,} \PY{n}{colm}\PY{p}{)}
         \PY{n}{ax}\PY{p}{[}\PY{l+m+mi}{0}\PY{p}{]}\PY{o}{.}\PY{n}{set\PYZus{}xlabel}\PY{p}{(}\PY{l+s}{\PYZsq{}}\PY{l+s}{height [km]}\PY{l+s}{\PYZsq{}}\PY{p}{)}
         \PY{n}{ax}\PY{p}{[}\PY{l+m+mi}{0}\PY{p}{]}\PY{o}{.}\PY{n}{set\PYZus{}ylabel}\PY{p}{(}\PY{l+s}{\PYZsq{}}\PY{l+s}{column mass [g cm\PYZca{}\PYZhy{}2]}\PY{l+s}{\PYZsq{}}\PY{p}{)}
         \PY{n}{ax}\PY{p}{[}\PY{l+m+mi}{1}\PY{p}{]}\PY{o}{.}\PY{n}{semilogy}\PY{p}{(}\PY{n}{h}\PY{p}{,} \PY{n}{colm}\PY{p}{)}
         \PY{n}{ax}\PY{p}{[}\PY{l+m+mi}{1}\PY{p}{]}\PY{o}{.}\PY{n}{set\PYZus{}xlabel}\PY{p}{(}\PY{l+s}{\PYZsq{}}\PY{l+s}{height [km]}\PY{l+s}{\PYZsq{}}\PY{p}{)}
         \PY{n}{fig}\PY{o}{.}\PY{n}{tight\PYZus{}layout}\PY{p}{(}\PY{p}{)}
         \PY{n}{plt}\PY{o}{.}\PY{n}{show}\PY{p}{(}\PY{p}{)}
\end{Verbatim}

    \begin{center}
    \adjustimage{max size={0.9\linewidth}{0.9\paperheight}}{LAB_B1_files/LAB_B1_17_0.png}
    \end{center}
    { \hspace*{\fill} \\}
    
    \textbf{Plot the gas density against height. Estimate the density scale
height \(H_{\rho}\) in \(\rho \approx \rho(0)exp(-h/H_{\rho})\) in the
photosphere.}

    \begin{Verbatim}[commandchars=\\\{\}]
{\color{incolor}In [{\color{incolor}14}]:} \PY{k}{def} \PY{n+nf}{rho}\PY{p}{(}\PY{n}{h}\PY{p}{,} \PY{n}{rho0}\PY{p}{,} \PY{n}{H}\PY{p}{)}\PY{p}{:}
             \PY{k}{return} \PY{n}{rho0}\PY{o}{*}\PY{n}{np}\PY{o}{.}\PY{n}{exp}\PY{p}{(}\PY{o}{\PYZhy{}}\PY{n}{h}\PY{o}{/}\PY{n}{H}\PY{p}{)}
         
         \PY{n}{plt}\PY{o}{.}\PY{n}{semilogy}\PY{p}{(}\PY{n}{h}\PY{p}{,} \PY{n}{dens}\PY{p}{)}
         \PY{n}{plt}\PY{o}{.}\PY{n}{semilogy}\PY{p}{(}\PY{n}{h}\PY{p}{,} \PY{n}{rho}\PY{p}{(}\PY{n}{h}\PY{p}{,} \PY{n}{dens}\PY{p}{[}\PY{l+m+mi}{79}\PY{p}{]}\PY{p}{,} \PY{l+m+mi}{125}\PY{p}{)}\PY{p}{)}
         \PY{n}{plt}\PY{o}{.}\PY{n}{text}\PY{p}{(}\PY{l+m+mi}{1500}\PY{p}{,}\PY{l+m+mf}{1e\PYZhy{}8}\PY{p}{,} \PY{l+s}{r\PYZdq{}}\PY{l+s}{\PYZdl{}H\PYZus{}\PYZob{}}\PY{l+s}{\PYZbs{}}\PY{l+s}{rho\PYZcb{} = 125\PYZdl{}}\PY{l+s}{\PYZdq{}}\PY{p}{)}
         \PY{n}{plt}\PY{o}{.}\PY{n}{ylabel}\PY{p}{(}\PY{l+s}{\PYZsq{}}\PY{l+s}{gas density [g cm\PYZdl{}\PYZca{}\PYZob{}\PYZhy{}3\PYZcb{}\PYZdl{}]}\PY{l+s}{\PYZsq{}}\PY{p}{)}
         \PY{n}{plt}\PY{o}{.}\PY{n}{xlabel}\PY{p}{(}\PY{l+s}{\PYZsq{}}\PY{l+s}{height [km]}\PY{l+s}{\PYZsq{}}\PY{p}{)}
         \PY{n}{plt}\PY{o}{.}\PY{n}{show}\PY{p}{(}\PY{p}{)}
\end{Verbatim}

    \begin{center}
    \adjustimage{max size={0.9\linewidth}{0.9\paperheight}}{LAB_B1_files/LAB_B1_19_0.png}
    \end{center}
    { \hspace*{\fill} \\}
    
    scale height \(H_{\rho} \approx 125\)

    \textbf{Compute the gas pressure and plot it against height. Overplot
the product \((nH+ne)kT\). Plot the ratio of the two curves to show
their differences. Do the differences measure deviations from the ideal
gas law or something else? Now add the helium density NHe to the product
and enlarge the deviations. Comments?}

The differences are not deviations from the ideal gas law, rather they
reflect the fact we arent including the contribution from He.

    \begin{Verbatim}[commandchars=\\\{\}]
{\color{incolor}In [{\color{incolor}15}]:} \PY{c}{\PYZsh{}\PYZdl{} P\PYZus{}\PYZob{}gas\PYZcb{} = P\PYZus{}\PYZob{}tot\PYZcb{} \PYZhy{} \PYZbs{}rho v\PYZus{}\PYZob{}t\PYZcb{}\PYZca{}2/2 \PYZdl{}}
         \PY{c}{\PYZsh{}pgas = ptot \PYZhy{} dens*vturb**2/2}
         \PY{n}{pgas} \PY{o}{=} \PY{n}{pgasptot}\PY{o}{*}\PY{n}{ptot}
         \PY{n}{plt}\PY{o}{.}\PY{n}{semilogy}\PY{p}{(}\PY{n}{h}\PY{p}{,} \PY{n}{pgas}\PY{p}{,} \PY{n}{label}\PY{o}{=}\PY{l+s}{r\PYZsq{}}\PY{l+s}{\PYZdl{}P\PYZus{}\PYZob{}gas\PYZcb{}\PYZdl{}}\PY{l+s}{\PYZsq{}}\PY{p}{)}
         \PY{n}{plt}\PY{o}{.}\PY{n}{semilogy}\PY{p}{(}\PY{n}{h}\PY{p}{,} \PY{p}{(}\PY{n}{nhyd}\PY{o}{+}\PY{n}{nel}\PY{p}{)}\PY{o}{*}\PY{n}{k}\PY{o}{*}\PY{n}{temp}\PY{p}{,} \PY{n}{label} \PY{o}{=} \PY{l+s}{r\PYZsq{}}\PY{l+s}{\PYZdl{}(nH+ne)kT\PYZdl{}}\PY{l+s}{\PYZsq{}} \PY{p}{)}
         \PY{n}{plt}\PY{o}{.}\PY{n}{xlabel}\PY{p}{(}\PY{l+s}{\PYZsq{}}\PY{l+s}{height [km]}\PY{l+s}{\PYZsq{}}\PY{p}{)}
         \PY{n}{plt}\PY{o}{.}\PY{n}{ylabel}\PY{p}{(}\PY{l+s}{\PYZsq{}}\PY{l+s}{pressure [dyn cm\PYZca{}\PYZhy{}2]}\PY{l+s}{\PYZsq{}}\PY{p}{)}
         
         \PY{n}{plt}\PY{o}{.}\PY{n}{legend}\PY{p}{(}\PY{p}{)}
         \PY{n}{plt}\PY{o}{.}\PY{n}{show}\PY{p}{(}\PY{p}{)}
\end{Verbatim}

    \begin{center}
    \adjustimage{max size={0.9\linewidth}{0.9\paperheight}}{LAB_B1_files/LAB_B1_22_0.png}
    \end{center}
    { \hspace*{\fill} \\}
    
    \begin{Verbatim}[commandchars=\\\{\}]
{\color{incolor}In [{\color{incolor}16}]:} \PY{n}{p2} \PY{o}{=} \PY{p}{(}\PY{n}{nhyd}\PY{o}{+}\PY{n}{nel}\PY{p}{)}\PY{o}{*}\PY{n}{k}\PY{o}{*}\PY{n}{temp}
         \PY{n}{p3} \PY{o}{=} \PY{p}{(}\PY{n}{nhyd}\PY{o}{+}\PY{n}{nel}\PY{o}{+}\PY{l+m+mf}{0.1}\PY{o}{*}\PY{n}{nhyd}\PY{p}{)}\PY{o}{*}\PY{n}{k}\PY{o}{*}\PY{n}{temp}
         
         \PY{n}{plt}\PY{o}{.}\PY{n}{plot}\PY{p}{(}\PY{n}{h}\PY{p}{,} \PY{n}{p2}\PY{o}{/}\PY{n}{pgas}\PY{p}{,} \PY{n}{label}\PY{o}{=}\PY{l+s}{r\PYZsq{}}\PY{l+s}{\PYZdl{}(n\PYZus{}H+n\PYZus{}e)kT/P\PYZus{}\PYZob{}gas\PYZcb{}\PYZdl{}}\PY{l+s}{\PYZsq{}}\PY{p}{)}
         \PY{n}{plt}\PY{o}{.}\PY{n}{xlabel}\PY{p}{(}\PY{l+s}{\PYZsq{}}\PY{l+s}{height [km]}\PY{l+s}{\PYZsq{}}\PY{p}{)}
         \PY{n}{plt}\PY{o}{.}\PY{n}{ylabel}\PY{p}{(}\PY{l+s}{\PYZsq{}}\PY{l+s}{ratio of pressures}\PY{l+s}{\PYZsq{}}\PY{p}{)}
         
         \PY{n}{plt}\PY{o}{.}\PY{n}{legend}\PY{p}{(}\PY{n}{loc}\PY{o}{=}\PY{l+s}{\PYZsq{}}\PY{l+s}{upper left}\PY{l+s}{\PYZsq{}}\PY{p}{)}
         \PY{n}{plt}\PY{o}{.}\PY{n}{show}\PY{p}{(}\PY{p}{)}
\end{Verbatim}

    \begin{center}
    \adjustimage{max size={0.9\linewidth}{0.9\paperheight}}{LAB_B1_files/LAB_B1_23_0.png}
    \end{center}
    { \hspace*{\fill} \\}
    
    \begin{Verbatim}[commandchars=\\\{\}]
{\color{incolor}In [{\color{incolor}17}]:} \PY{n}{p3} \PY{o}{=} \PY{p}{(}\PY{n}{nhyd} \PY{o}{+} \PY{n}{nel} \PY{o}{+} \PY{l+m+mf}{0.1}\PY{o}{*}\PY{n}{nhyd}\PY{p}{)}\PY{o}{*}\PY{n}{k}\PY{o}{*}\PY{n}{temp}
         \PY{n}{plt}\PY{o}{.}\PY{n}{plot}\PY{p}{(}\PY{n}{h}\PY{p}{,} \PY{n}{p3}\PY{o}{/}\PY{n}{pgas}\PY{o}{\PYZhy{}}\PY{l+m+mi}{1}\PY{p}{,} \PY{n}{label}\PY{o}{=}\PY{l+s}{r\PYZsq{}}\PY{l+s}{\PYZdl{}(nH+nHe+ne)kT/P\PYZus{}\PYZob{}gas\PYZcb{}\PYZhy{}1 \PYZdl{}}\PY{l+s}{\PYZsq{}}\PY{p}{)}
         
         \PY{n}{plt}\PY{o}{.}\PY{n}{xlabel}\PY{p}{(}\PY{l+s}{\PYZsq{}}\PY{l+s}{height [km]}\PY{l+s}{\PYZsq{}}\PY{p}{)}
         \PY{n}{plt}\PY{o}{.}\PY{n}{ylabel}\PY{p}{(}\PY{l+s}{\PYZsq{}}\PY{l+s}{ratio of pressures}\PY{l+s}{\PYZsq{}}\PY{p}{)}
         
         \PY{n}{plt}\PY{o}{.}\PY{n}{legend}\PY{p}{(}\PY{n}{loc}\PY{o}{=}\PY{l+s}{\PYZsq{}}\PY{l+s}{upper left}\PY{l+s}{\PYZsq{}}\PY{p}{)}
         \PY{n}{plt}\PY{o}{.}\PY{n}{show}\PY{p}{(}\PY{p}{)}
\end{Verbatim}

    \begin{center}
    \adjustimage{max size={0.9\linewidth}{0.9\paperheight}}{LAB_B1_files/LAB_B1_24_0.png}
    \end{center}
    { \hspace*{\fill} \\}
    
    \textbf{Comments?}

The addition of the He density puts the gas pressure in agreement with
the ideal gas law.

    \textbf{Plot the total hydrogen density against height and overplot
curves for the electron density, the proton density, and the density of
the electrons that do not result from hydrogen ionization. Explain their
behavior. You may find inspiration in Figure 6 on page 13. The last
curve is parallel to the hydrogen density over a considerable height
range. What does that imply? And what happens at larger height?}

    \begin{Verbatim}[commandchars=\\\{\}]
{\color{incolor}In [{\color{incolor}18}]:} \PY{n}{plt}\PY{o}{.}\PY{n}{semilogy}\PY{p}{(}\PY{n}{h}\PY{p}{,} \PY{n}{nhyd}\PY{p}{,} \PY{n}{label}\PY{o}{=}\PY{l+s}{r\PYZsq{}}\PY{l+s}{\PYZdl{}n\PYZus{}H\PYZdl{}}\PY{l+s}{\PYZsq{}}\PY{p}{)}
         \PY{n}{plt}\PY{o}{.}\PY{n}{semilogy}\PY{p}{(}\PY{n}{h}\PY{p}{,} \PY{n}{nel}\PY{p}{,} \PY{n}{label}\PY{o}{=}\PY{l+s}{r\PYZsq{}}\PY{l+s}{\PYZdl{}n\PYZus{}e\PYZdl{}}\PY{l+s}{\PYZsq{}}\PY{p}{)}
         \PY{n}{plt}\PY{o}{.}\PY{n}{semilogy}\PY{p}{(}\PY{n}{h}\PY{p}{,} \PY{n}{nprot}\PY{p}{,} \PY{n}{label}\PY{o}{=}\PY{l+s}{r\PYZsq{}}\PY{l+s}{\PYZdl{}n\PYZus{}p\PYZdl{}}\PY{l+s}{\PYZsq{}}\PY{p}{)}
         \PY{n}{n\PYZus{}metal} \PY{o}{=} \PY{n}{nel} \PY{o}{\PYZhy{}} \PY{n}{nprot}
         \PY{n}{plt}\PY{o}{.}\PY{n}{semilogy}\PY{p}{(}\PY{n}{h}\PY{p}{,} \PY{n}{n\PYZus{}metal}\PY{p}{,} \PY{n}{label}\PY{o}{=}\PY{l+s}{r\PYZsq{}}\PY{l+s}{\PYZdl{}n\PYZus{}\PYZob{}metal e\PYZcb{}\PYZdl{}}\PY{l+s}{\PYZsq{}}\PY{p}{)}
         \PY{n}{plt}\PY{o}{.}\PY{n}{legend}\PY{p}{(}\PY{p}{)}
         \PY{n}{plt}\PY{o}{.}\PY{n}{show}\PY{p}{(}\PY{p}{)}
\end{Verbatim}

    \begin{center}
    \adjustimage{max size={0.9\linewidth}{0.9\paperheight}}{LAB_B1_files/LAB_B1_27_0.png}
    \end{center}
    { \hspace*{\fill} \\}
    
    The \(n_p\) traces the temperature as it decreases then inverts. From
\(h=0\) until the inversion, \(n_e\) is dominated by electrons coming
from ionized metals. At the inversion, more H becomes ionized and this
becomes the dominant electron source. The density of electrons due to
metal ionization paralles the density of hydrogren since \(n_H\) traces
the density of metals through out the atmosphere, which exponentially
decreases with height. At larger heights up in the atmosphere the
temperature rises enough to further ionization of the metals,
increaseing the number of electrons (compared to the number of metal
ions) and flattening out the \(n_{metal e^-}\)curve.

    \textbf{Plot the ionization fraction of hydrogen logarithmically against
height. Why does this curve look like the one in Figure 2? And why is it
tilted with respect to that?}

    \begin{Verbatim}[commandchars=\\\{\}]
{\color{incolor}In [{\color{incolor}19}]:} \PY{n}{N} \PY{o}{=} \PY{n}{nprot} \PY{o}{+} \PY{n}{nel} \PY{o}{\PYZhy{}} \PY{n}{n\PYZus{}metal}
         \PY{n}{plt}\PY{o}{.}\PY{n}{semilogy}\PY{p}{(}\PY{n}{h}\PY{p}{,} \PY{n}{N}\PY{o}{/}\PY{n}{nhyd}\PY{p}{,} \PY{n}{label}\PY{o}{=}\PY{l+s}{r\PYZsq{}}\PY{l+s}{\PYZdl{}n\PYZus{}H\PYZdl{}}\PY{l+s}{\PYZsq{}}\PY{p}{)}
         \PY{n}{plt}\PY{o}{.}\PY{n}{xlabel}\PY{p}{(}\PY{l+s}{\PYZsq{}}\PY{l+s}{height [km]}\PY{l+s}{\PYZsq{}}\PY{p}{)}
         \PY{n}{plt}\PY{o}{.}\PY{n}{ylabel}\PY{p}{(}\PY{l+s}{\PYZsq{}}\PY{l+s}{\PYZdl{}N/N\PYZus{}H\PYZdl{}}\PY{l+s}{\PYZsq{}}\PY{p}{)}
         \PY{n}{plt}\PY{o}{.}\PY{n}{show}\PY{p}{(}\PY{p}{)}
\end{Verbatim}

    \begin{center}
    \adjustimage{max size={0.9\linewidth}{0.9\paperheight}}{LAB_B1_files/LAB_B1_30_0.png}
    \end{center}
    { \hspace*{\fill} \\}
    
    The ionization fraction mirrors the temperature stratification since the
temperature sets the ionization rate given by the Saha equation which
scales as \(T^{3/2}e^{-T}\). This follows T until very high temperatures
in the outer atmosphere where the exponential takes over and the
ionization fraction diverges from the temperature curve.

    ** Let us now compare the photon and particle densities. In
thermodynamic equilibrium (TE) the radiation is isotropic with intensity
\(I_{\nu} = B_{\nu}\) and has total energy density (Stefan Boltzmann)**

\[u = \frac{1}{c} \iint B_{\nu}d\Omega d\nu = \frac{4\sigma}{c} T^4 \]

\textbf{so that the total photon density for isotropic TE radiation is
given, with \(u_{\nu} = du/d\nu\), \(T\) in K and \(N_{phot}\) in
photons per cm\(^3\), by}

\[ N_{phot} = \int_{0}^{\infty} \frac{u_{\nu}}{h\nu}d\nu \approx 20T^3 \]

\textbf{This equation gives a reasonable estimate for the photon density
at the deepest model location, why?}

We are assuming the deepest levels are in LTE since the density is high
enough for collisions to couple photons and Hydrogen.

\textbf{Compute the value there and compare it to the hydrogen density.}

\(N_{phot} = 1.6 \times 10^{13}\) and \(N_{phot}/N_{H} = 0.00012\)

    \begin{Verbatim}[commandchars=\\\{\}]
{\color{incolor}In [{\color{incolor}20}]:} \PY{n}{n} \PY{o}{=} \PY{l+m+mi}{20}\PY{o}{*}\PY{n}{temp}\PY{p}{[}\PY{o}{\PYZhy{}}\PY{l+m+mi}{1}\PY{p}{]}\PY{o}{*}\PY{o}{*}\PY{l+m+mi}{3}
         \PY{c}{\PYZsh{}print n, nhyd[\PYZhy{}1], temp[\PYZhy{}1], temp[0]}
         \PY{k}{print} \PY{l+s}{\PYZdq{}}\PY{l+s}{\PYZob{}:.2\PYZcb{}}\PY{l+s}{\PYZdq{}}\PY{o}{.}\PY{n}{format}\PY{p}{(}\PY{n}{n}\PY{o}{/}\PY{n}{nhyd}\PY{p}{[}\PY{o}{\PYZhy{}}\PY{l+m+mi}{1}\PY{p}{]}\PY{p}{)}
         
         \PY{n}{natmo} \PY{o}{=} \PY{l+m+mi}{20}\PY{o}{*}\PY{l+m+mi}{5770}\PY{o}{*}\PY{o}{*}\PY{l+m+mi}{3}\PY{o}{/}\PY{p}{(}\PY{l+m+mi}{2}\PY{o}{*}\PY{n}{np}\PY{o}{.}\PY{n}{pi}\PY{p}{)}
         
         \PY{k}{print} \PY{n}{natmo}\PY{p}{,} \PY{n}{natmo}\PY{o}{/}\PY{n}{nhyd}\PY{p}{[}\PY{l+m+mi}{0}\PY{p}{]}
\end{Verbatim}

    \begin{Verbatim}[commandchars=\\\{\}]
0.00012
6.11473396401e+11 109.681326709
    \end{Verbatim}

    \textbf{Why is the equation not valid higher up in the atmosphere?} It
is not valid higher up in the atmosphere because the density is low
enough that we are in the NLTE regime.

\textbf{The photon density there is
\(N_{phot} \approx 20T_{eff}^{3}/2\pi\) with \(T_{eff} = 5770\)K the
effective solar temperature (since
\(\pi B(T_{eff}) = \sigma T_{eff}^3 = \mathscr{F}^{+} = \pi \overline{I^{+}}\)
with \(\mathscr{F}^{+}\) the emergeng flux and \(\overline{I^{+}}\) the
disk averaged emergent intensity). Compare it to the hydrogen density at
the highest location in the FALC model. The medium there is insensitive
to these photons (except those at the center wavelength of the hydrogen
Ly \(\alpha\) line), why?}

\(N_{phot} = 6 \times 10^{11}\) and \(N_{phot}/N_{H} = 110\)

The medium is insensitive to these photons since the optical depth
outside of line center is \(<<1\) (the pressure is low so there is no
pressure broadening) and thus we are in NLTE due to the low density at
these altitudes.

    \subsection*{1.3 Comparison with the earth's
atmosphere}\label{comparison-with-the-earths-atmosphere}

\textbf{Write IDL code to read file earth.dat.}

    \begin{Verbatim}[commandchars=\\\{\}]
{\color{incolor}In [{\color{incolor}21}]:} \PY{o}{!}head earth.dat
\end{Verbatim}

    \begin{Verbatim}[commandchars=\\\{\}]
0   6.01   288   -2.91   19.41
   1   5.95   282   -2.95   19.36
   2   5.90   275   -3.00   19.31
   3   5.85   269   -3.04   19.28 
   4   5.79   262   -3.09   19.23
   5   5.73   256   -3.13   19.19
   6   5.67   249   -3.18   19.14

   8   5.55   236   -3.28   19.04
    \end{Verbatim}

    \begin{Verbatim}[commandchars=\\\{\}]
{\color{incolor}In [{\color{incolor}22}]:} \PY{n}{h}\PY{p}{,} \PY{n}{logP}\PY{p}{,} \PY{n}{T}\PY{p}{,} \PY{n}{logdens}\PY{p}{,} \PY{n}{logN} \PY{o}{=} \PY{n}{np}\PY{o}{.}\PY{n}{loadtxt}\PY{p}{(}\PY{l+s}{\PYZsq{}}\PY{l+s}{earth.dat}\PY{l+s}{\PYZsq{}}\PY{p}{,} \PY{n}{unpack}\PY{o}{=}\PY{n+nb+bp}{True}\PY{p}{)}
\end{Verbatim}

    \textbf{Plot the temperature, pressure, particle density and gas density
against height, logarithmically where appropriate.}

    \begin{Verbatim}[commandchars=\\\{\}]
{\color{incolor}In [{\color{incolor}23}]:} \PY{n}{plt}\PY{o}{.}\PY{n}{plot}\PY{p}{(}\PY{n}{h}\PY{p}{,} \PY{n}{T}\PY{p}{)}
         \PY{n}{plt}\PY{o}{.}\PY{n}{ylabel}\PY{p}{(}\PY{l+s}{\PYZsq{}}\PY{l+s}{T [K]}\PY{l+s}{\PYZsq{}}\PY{p}{)}
         \PY{n}{plt}\PY{o}{.}\PY{n}{xlabel}\PY{p}{(}\PY{l+s}{\PYZsq{}}\PY{l+s}{height [km]}\PY{l+s}{\PYZsq{}}\PY{p}{)}
         \PY{n}{plt}\PY{o}{.}\PY{n}{title}\PY{p}{(}\PY{l+s}{\PYZsq{}}\PY{l+s}{Temperature v height}\PY{l+s}{\PYZsq{}}\PY{p}{)}
         \PY{n}{plt}\PY{o}{.}\PY{n}{text}\PY{p}{(}\PY{l+m+mi}{25}\PY{p}{,} \PY{l+m+mi}{1200}\PY{p}{,} \PY{l+s}{\PYZsq{}}\PY{l+s}{EARTH ATMOSPHERE}\PY{l+s}{\PYZsq{}}\PY{p}{)}
         \PY{n}{plt}\PY{o}{.}\PY{n}{show}\PY{p}{(}\PY{p}{)}
\end{Verbatim}

    \begin{center}
    \adjustimage{max size={0.9\linewidth}{0.9\paperheight}}{LAB_B1_files/LAB_B1_39_0.png}
    \end{center}
    { \hspace*{\fill} \\}
    
    \begin{Verbatim}[commandchars=\\\{\}]
{\color{incolor}In [{\color{incolor}24}]:} \PY{n}{P} \PY{o}{=} \PY{l+m+mi}{10}\PY{o}{*}\PY{o}{*}\PY{n}{logP}
         \PY{n}{plt}\PY{o}{.}\PY{n}{semilogy}\PY{p}{(}\PY{n}{h}\PY{p}{,} \PY{n}{P}\PY{p}{)}
         \PY{n}{plt}\PY{o}{.}\PY{n}{ylabel}\PY{p}{(}\PY{l+s}{\PYZsq{}}\PY{l+s}{P [dyn cm\PYZdl{}\PYZca{}\PYZob{}\PYZhy{}2\PYZcb{}\PYZdl{}]}\PY{l+s}{\PYZsq{}}\PY{p}{)}
         \PY{n}{plt}\PY{o}{.}\PY{n}{title}\PY{p}{(}\PY{l+s}{\PYZsq{}}\PY{l+s}{Pressure v height}\PY{l+s}{\PYZsq{}}\PY{p}{)}
         \PY{n}{plt}\PY{o}{.}\PY{n}{xlabel}\PY{p}{(}\PY{l+s}{\PYZsq{}}\PY{l+s}{height [km]}\PY{l+s}{\PYZsq{}}\PY{p}{)}
         \PY{n}{plt}\PY{o}{.}\PY{n}{show}\PY{p}{(}\PY{p}{)}
\end{Verbatim}

    \begin{center}
    \adjustimage{max size={0.9\linewidth}{0.9\paperheight}}{LAB_B1_files/LAB_B1_40_0.png}
    \end{center}
    { \hspace*{\fill} \\}
    
    \begin{Verbatim}[commandchars=\\\{\}]
{\color{incolor}In [{\color{incolor}25}]:} \PY{n}{N} \PY{o}{=} \PY{l+m+mi}{10}\PY{o}{*}\PY{o}{*}\PY{n}{logN}
         \PY{n}{plt}\PY{o}{.}\PY{n}{semilogy}\PY{p}{(}\PY{n}{h}\PY{p}{,} \PY{n}{N}\PY{p}{)}
         \PY{n}{plt}\PY{o}{.}\PY{n}{ylabel}\PY{p}{(}\PY{l+s}{\PYZsq{}}\PY{l+s}{N [cm\PYZdl{}\PYZca{}\PYZob{}\PYZhy{}3\PYZcb{}\PYZdl{}]}\PY{l+s}{\PYZsq{}}\PY{p}{)}
         \PY{n}{plt}\PY{o}{.}\PY{n}{title}\PY{p}{(}\PY{l+s}{\PYZsq{}}\PY{l+s}{particle density v height}\PY{l+s}{\PYZsq{}}\PY{p}{)}
         \PY{n}{plt}\PY{o}{.}\PY{n}{xlabel}\PY{p}{(}\PY{l+s}{\PYZsq{}}\PY{l+s}{height [km]}\PY{l+s}{\PYZsq{}}\PY{p}{)}
         \PY{n}{plt}\PY{o}{.}\PY{n}{show}\PY{p}{(}\PY{p}{)}
\end{Verbatim}

    \begin{center}
    \adjustimage{max size={0.9\linewidth}{0.9\paperheight}}{LAB_B1_files/LAB_B1_41_0.png}
    \end{center}
    { \hspace*{\fill} \\}
    
    \begin{Verbatim}[commandchars=\\\{\}]
{\color{incolor}In [{\color{incolor}26}]:} \PY{n}{dens} \PY{o}{=} \PY{l+m+mi}{10}\PY{o}{*}\PY{o}{*}\PY{n}{logdens}
         \PY{n}{plt}\PY{o}{.}\PY{n}{semilogy}\PY{p}{(}\PY{n}{h}\PY{p}{,} \PY{n}{dens}\PY{p}{)}
         \PY{n}{plt}\PY{o}{.}\PY{n}{ylabel}\PY{p}{(}\PY{l+s}{r\PYZsq{}}\PY{l+s}{\PYZdl{}}\PY{l+s}{\PYZbs{}}\PY{l+s}{rho\PYZdl{} [g cm\PYZdl{}\PYZca{}\PYZob{}\PYZhy{}3\PYZcb{}\PYZdl{}]}\PY{l+s}{\PYZsq{}}\PY{p}{)}
         \PY{n}{plt}\PY{o}{.}\PY{n}{title}\PY{p}{(}\PY{l+s}{\PYZsq{}}\PY{l+s}{gas density v height}\PY{l+s}{\PYZsq{}}\PY{p}{)}
         \PY{n}{plt}\PY{o}{.}\PY{n}{xlabel}\PY{p}{(}\PY{l+s}{\PYZsq{}}\PY{l+s}{height [km]}\PY{l+s}{\PYZsq{}}\PY{p}{)}
         \PY{n}{plt}\PY{o}{.}\PY{n}{show}\PY{p}{(}\PY{p}{)}
\end{Verbatim}

    \begin{center}
    \adjustimage{max size={0.9\linewidth}{0.9\paperheight}}{LAB_B1_files/LAB_B1_42_0.png}
    \end{center}
    { \hspace*{\fill} \\}
    
    \textbf{Plot the pressure and density stratifications together in
normalized units in one graph. Comments?}

    \begin{Verbatim}[commandchars=\\\{\}]
{\color{incolor}In [{\color{incolor}27}]:} \PY{c}{\PYZsh{} normalze:}
         
         \PY{n}{norm} \PY{o}{=} \PY{p}{(}\PY{n}{logP}\PY{p}{[}\PY{l+m+mi}{0}\PY{p}{]} \PY{o}{+} \PY{n}{logdens}\PY{p}{[}\PY{l+m+mi}{0}\PY{p}{]}\PY{p}{)}
         
         
         \PY{n}{plt}\PY{o}{.}\PY{n}{plot}\PY{p}{(}\PY{n}{h}\PY{p}{,} \PY{n}{logdens}\PY{o}{/}\PY{n}{norm}\PY{o}{+}\PY{l+m+mf}{0.9}\PY{p}{,} \PY{n}{label}\PY{o}{=}\PY{l+s}{\PYZsq{}}\PY{l+s}{density}\PY{l+s}{\PYZsq{}}\PY{p}{)}
         \PY{n}{plt}\PY{o}{.}\PY{n}{plot}\PY{p}{(}\PY{n}{h}\PY{p}{,} \PY{n}{logP}\PY{o}{/}\PY{n}{norm}\PY{o}{\PYZhy{}}\PY{l+m+mf}{2.0}\PY{p}{,} \PY{n}{label} \PY{o}{=} \PY{l+s}{\PYZsq{}}\PY{l+s}{pressure}\PY{l+s}{\PYZsq{}}\PY{p}{)}
         \PY{n}{plt}\PY{o}{.}\PY{n}{ylabel}\PY{p}{(}\PY{l+s}{r\PYZsq{}}\PY{l+s}{density, pressure [normalized]}\PY{l+s}{\PYZsq{}}\PY{p}{)}
         \PY{n}{plt}\PY{o}{.}\PY{n}{title}\PY{p}{(}\PY{l+s}{\PYZsq{}}\PY{l+s}{normalized stratification v height}\PY{l+s}{\PYZsq{}}\PY{p}{)}
         \PY{n}{plt}\PY{o}{.}\PY{n}{xlabel}\PY{p}{(}\PY{l+s}{\PYZsq{}}\PY{l+s}{height [km]}\PY{l+s}{\PYZsq{}}\PY{p}{)}
         \PY{n}{plt}\PY{o}{.}\PY{n}{legend}\PY{p}{(}\PY{p}{)}
         \PY{n}{plt}\PY{o}{.}\PY{n}{show}\PY{p}{(}\PY{p}{)}
\end{Verbatim}

    \begin{center}
    \adjustimage{max size={0.9\linewidth}{0.9\paperheight}}{LAB_B1_files/LAB_B1_44_0.png}
    \end{center}
    { \hspace*{\fill} \\}
    
    The pressure and density follow each other for most of the height of the
atmosphere. However, they depart at around heights around 120km. This
departure must be due to a difference in species at higher heights
acording to the ideal gas law, \(P = (N/V)kT = (\rho/\mu)kT\).

    \textbf{Plot the mean molecular weight
\(\mu_E = \overline{m}/m_H = \rho/(N m_H)\) against height. Why does it
decrease in the high atmosphere?}

    \begin{Verbatim}[commandchars=\\\{\}]
{\color{incolor}In [{\color{incolor}28}]:} \PY{c}{\PYZsh{} h, logP, T, logdens, logN}
         
         \PY{n}{dens} \PY{o}{=} \PY{l+m+mi}{10}\PY{o}{*}\PY{o}{*}\PY{p}{(}\PY{n}{logdens}\PY{p}{)}
         \PY{n}{N} \PY{o}{=} \PY{l+m+mi}{10}\PY{o}{*}\PY{o}{*}\PY{p}{(}\PY{n}{logN}\PY{p}{)}
         
         \PY{n}{mu} \PY{o}{=} \PY{n}{dens}\PY{o}{/}\PY{p}{(}\PY{n}{N}\PY{o}{*}\PY{n}{mh}\PY{p}{)}
\end{Verbatim}

    \begin{Verbatim}[commandchars=\\\{\}]
{\color{incolor}In [{\color{incolor}29}]:} \PY{n}{plt}\PY{o}{.}\PY{n}{plot}\PY{p}{(}\PY{n}{h}\PY{p}{,} \PY{n}{mu}\PY{p}{)}
         \PY{n}{plt}\PY{o}{.}\PY{n}{ylabel}\PY{p}{(}\PY{l+s}{r\PYZsq{}}\PY{l+s}{\PYZdl{}}\PY{l+s}{\PYZbs{}}\PY{l+s}{mu\PYZdl{}}\PY{l+s}{\PYZsq{}}\PY{p}{)}
         \PY{n}{plt}\PY{o}{.}\PY{n}{title}\PY{p}{(}\PY{l+s}{\PYZsq{}}\PY{l+s}{mean molecular weight per particle}\PY{l+s}{\PYZsq{}}\PY{p}{)}
         \PY{n}{plt}\PY{o}{.}\PY{n}{xlabel}\PY{p}{(}\PY{l+s}{\PYZsq{}}\PY{l+s}{height [km]}\PY{l+s}{\PYZsq{}}\PY{p}{)}
         \PY{n}{plt}\PY{o}{.}\PY{n}{show}\PY{p}{(}\PY{p}{)}
\end{Verbatim}

    \begin{center}
    \adjustimage{max size={0.9\linewidth}{0.9\paperheight}}{LAB_B1_files/LAB_B1_48_0.png}
    \end{center}
    { \hspace*{\fill} \\}
    
    The mean molecular weight decreases high in the atmosphere because
heavier molecules segregate to the lower atmosphere as only the lighter
molecules have enough kinetic energy to reach those heights. In
addition, as the height increases molecules may become dissasociated due
to UV dissociation.

    \textbf{Estimate the density scales height of the lower terrestrial
atmosphere. Which quantities make it differ from the solar one? How much
harder do you have to breathe on Mount Everest?}

The density profile makes the earth scale height differ from the solar
scale height.

    \begin{Verbatim}[commandchars=\\\{\}]
{\color{incolor}In [{\color{incolor}30}]:} \PY{n}{plt}\PY{o}{.}\PY{n}{semilogy}\PY{p}{(}\PY{n}{h}\PY{p}{,} \PY{n}{dens}\PY{p}{)}
         \PY{n}{plt}\PY{o}{.}\PY{n}{semilogy}\PY{p}{(}\PY{n}{h}\PY{p}{,} \PY{n}{rho}\PY{p}{(}\PY{n}{h}\PY{p}{,} \PY{n}{dens}\PY{p}{[}\PY{l+m+mi}{0}\PY{p}{]}\PY{p}{,} \PY{l+m+mi}{7}\PY{p}{)}\PY{p}{)}
         \PY{n}{plt}\PY{o}{.}\PY{n}{ylabel}\PY{p}{(}\PY{l+s}{r\PYZsq{}}\PY{l+s}{\PYZdl{}}\PY{l+s}{\PYZbs{}}\PY{l+s}{rho\PYZdl{} [g cm\PYZdl{}\PYZca{}\PYZob{}\PYZhy{}3\PYZcb{}\PYZdl{}]}\PY{l+s}{\PYZsq{}}\PY{p}{)}
         \PY{n}{plt}\PY{o}{.}\PY{n}{title}\PY{p}{(}\PY{l+s}{\PYZsq{}}\PY{l+s}{gas density v height}\PY{l+s}{\PYZsq{}}\PY{p}{)}
         \PY{n}{plt}\PY{o}{.}\PY{n}{xlabel}\PY{p}{(}\PY{l+s}{\PYZsq{}}\PY{l+s}{height [km]}\PY{l+s}{\PYZsq{}}\PY{p}{)}
         \PY{n}{plt}\PY{o}{.}\PY{n}{ylim}\PY{p}{(}\PY{l+m+mf}{1e\PYZhy{}14}\PY{p}{,} \PY{l+m+mf}{1e\PYZhy{}2}\PY{p}{)}
         \PY{n}{plt}\PY{o}{.}\PY{n}{text}\PY{p}{(}\PY{l+m+mi}{200}\PY{p}{,}\PY{l+m+mf}{1e\PYZhy{}6}\PY{p}{,} \PY{l+s}{r\PYZdq{}}\PY{l+s}{\PYZdl{}H\PYZus{}\PYZob{}}\PY{l+s}{\PYZbs{}}\PY{l+s}{rho\PYZcb{} = \PYZdl{}7 km}\PY{l+s}{\PYZdq{}}\PY{p}{)}
         \PY{n}{plt}\PY{o}{.}\PY{n}{show}\PY{p}{(}\PY{p}{)}
\end{Verbatim}

    \begin{center}
    \adjustimage{max size={0.9\linewidth}{0.9\paperheight}}{LAB_B1_files/LAB_B1_51_0.png}
    \end{center}
    { \hspace*{\fill} \\}
    
    \begin{Verbatim}[commandchars=\\\{\}]
{\color{incolor}In [{\color{incolor}31}]:} \PY{k}{print} \PY{l+s}{\PYZdq{}}\PY{l+s}{density at summit}\PY{l+s}{\PYZdq{}}\PY{p}{,}\PY{n}{dens}\PY{p}{[}\PY{l+m+mi}{8}\PY{p}{]} \PY{o}{/} \PY{n}{dens}\PY{p}{[}\PY{l+m+mi}{0}\PY{p}{]}
         \PY{k}{print} \PY{l+s}{\PYZdq{}}\PY{l+s}{pressure at summit}\PY{l+s}{\PYZdq{}}\PY{p}{,}\PY{l+m+mi}{10}\PY{o}{*}\PY{o}{*}\PY{n}{logP}\PY{p}{[}\PY{l+m+mi}{8}\PY{p}{]} \PY{o}{/} \PY{l+m+mi}{10}\PY{o}{*}\PY{o}{*}\PY{n}{logP}\PY{p}{[}\PY{l+m+mi}{0}\PY{p}{]}
\end{Verbatim}

    \begin{Verbatim}[commandchars=\\\{\}]
density at summit 0.338844156139
pressure at summit 0.257039578277
    \end{Verbatim}

    Everest is \textasciitilde{}9000m \textasciitilde{}10km so the gas
density is \textasciitilde{}1/3 (slightly less than \(1/e\)) the density
at sea level while the pressure is 1/4 that of the surface. Therefore
one would have to breath \textasciitilde{}3 times harder to get the same
amount of oxygen.

    \textbf{Compare the terrestrial parameter values to the solar ones, at
the base of each atmosphere. What is the ratio of the particle densities
at h = 0 in the two atmospheres?}

    \begin{Verbatim}[commandchars=\\\{\}]
{\color{incolor}In [{\color{incolor}32}]:} \PY{c}{\PYZsh{} terrestrial: h, logP, T, logdens, logN}
         \PY{c}{\PYZsh{} solar: hsun, ptot, temp, dens, nhyd}
\end{Verbatim}

    \begin{Verbatim}[commandchars=\\\{\}]
{\color{incolor}In [{\color{incolor}33}]:} \PY{k}{print} \PY{l+s}{\PYZdq{}}\PY{l+s}{      solar }\PY{l+s+se}{\PYZbs{}t}\PY{l+s}{ earth}\PY{l+s}{\PYZdq{}}
         \PY{k}{print} \PY{l+s}{\PYZdq{}}\PY{l+s}{h     \PYZob{}:.2\PYZcb{} }\PY{l+s+se}{\PYZbs{}t}\PY{l+s}{ \PYZob{}:.2\PYZcb{}}\PY{l+s}{\PYZdq{}}\PY{o}{.}\PY{n}{format}\PY{p}{(}\PY{n}{hsun}\PY{p}{[}\PY{o}{\PYZhy{}}\PY{l+m+mi}{11}\PY{p}{]}\PY{p}{,} \PY{n}{h}\PY{p}{[}\PY{l+m+mi}{0}\PY{p}{]}\PY{p}{)}
         \PY{k}{print} \PY{l+s}{\PYZdq{}}\PY{l+s}{press \PYZob{}:.2\PYZcb{} }\PY{l+s+se}{\PYZbs{}t}\PY{l+s}{ \PYZob{}:.2\PYZcb{}}\PY{l+s}{\PYZdq{}}\PY{o}{.}\PY{n}{format}\PY{p}{(}\PY{n}{ptot}\PY{p}{[}\PY{o}{\PYZhy{}}\PY{l+m+mi}{11}\PY{p}{]}\PY{p}{,} \PY{l+m+mi}{10}\PY{o}{*}\PY{o}{*}\PY{n}{logP}\PY{p}{[}\PY{l+m+mi}{0}\PY{p}{]}\PY{p}{)}
         \PY{k}{print} \PY{l+s}{\PYZdq{}}\PY{l+s}{temp  \PYZob{}:.2\PYZcb{} }\PY{l+s+se}{\PYZbs{}t}\PY{l+s}{ \PYZob{}:.2\PYZcb{}}\PY{l+s}{\PYZdq{}}\PY{o}{.}\PY{n}{format}\PY{p}{(}\PY{n}{temp}\PY{p}{[}\PY{o}{\PYZhy{}}\PY{l+m+mi}{11}\PY{p}{]}\PY{p}{,} \PY{n}{T}\PY{p}{[}\PY{l+m+mi}{0}\PY{p}{]}\PY{p}{)}
         \PY{k}{print} \PY{l+s}{\PYZdq{}}\PY{l+s}{dens  \PYZob{}:.2\PYZcb{} }\PY{l+s+se}{\PYZbs{}t}\PY{l+s}{ \PYZob{}:.2\PYZcb{}}\PY{l+s}{\PYZdq{}}\PY{o}{.}\PY{n}{format}\PY{p}{(}\PY{n}{dens}\PY{p}{[}\PY{o}{\PYZhy{}}\PY{l+m+mi}{11}\PY{p}{]}\PY{p}{,} \PY{l+m+mi}{10}\PY{o}{*}\PY{o}{*}\PY{n}{logdens}\PY{p}{[}\PY{l+m+mi}{0}\PY{p}{]}\PY{p}{)}
         \PY{k}{print} \PY{l+s}{\PYZdq{}}\PY{l+s}{N     \PYZob{}:.2\PYZcb{} }\PY{l+s+se}{\PYZbs{}t}\PY{l+s}{ \PYZob{}:.2\PYZcb{}}\PY{l+s}{\PYZdq{}}\PY{o}{.}\PY{n}{format}\PY{p}{(}\PY{n}{nhyd}\PY{p}{[}\PY{o}{\PYZhy{}}\PY{l+m+mi}{11}\PY{p}{]}\PY{p}{,} \PY{l+m+mi}{10}\PY{o}{*}\PY{o}{*}\PY{n}{logN}\PY{p}{[}\PY{l+m+mi}{0}\PY{p}{]}\PY{p}{)}
         \PY{k}{print} \PY{l+s}{\PYZsq{}}\PY{l+s+se}{\PYZbs{}n}\PY{l+s}{\PYZsq{}}
         \PY{k}{print} \PY{l+s}{\PYZdq{}}\PY{l+s}{ratio of particle density N\PYZus{}earth/N\PYZus{}sun \PYZob{}:.4\PYZcb{}}\PY{l+s}{\PYZdq{}}\PY{o}{.}\PY{n}{format}\PY{p}{(}\PY{l+m+mi}{10}\PY{o}{*}\PY{o}{*}\PY{n}{logN}\PY{p}{[}\PY{l+m+mi}{0}\PY{p}{]}\PY{o}{/}\PY{n}{nhyd}\PY{p}{[}\PY{o}{\PYZhy{}}\PY{l+m+mi}{11}\PY{p}{]}\PY{p}{)}
\end{Verbatim}

    \begin{Verbatim}[commandchars=\\\{\}]
solar 	 earth
h     0.0 	 0.0
press 1.2e+05 	 1e+06
temp  6.5e+03 	 2.9e+02
dens  3.2e-07 	 0.0012
N     1.2e+17 	 2.6e+19


ratio of particle density N\_earth/N\_sun 217.5
    \end{Verbatim}

    \begin{Verbatim}[commandchars=\\\{\}]
{\color{incolor}In [{\color{incolor}34}]:} \PY{n}{mcol\PYZus{}earth} \PY{o}{=} \PY{l+m+mi}{10}\PY{o}{*}\PY{o}{*}\PY{n}{logP}\PY{p}{[}\PY{l+m+mi}{0}\PY{p}{]}\PY{o}{/}\PY{l+m+mf}{980.665}
         \PY{k}{print} \PY{n}{mcol\PYZus{}earth}
\end{Verbatim}

    \begin{Verbatim}[commandchars=\\\{\}]
1043.46845486
    \end{Verbatim}

    \textbf{The standard gravity at the earth's surface is \(g_E = 980.665\)
cm s\(^{−2}\). Use this value to estimate the atmospheric column mass (g
cm\(^{−2}\)) at the earth's surface and compare that also to the value
at the base of the solar atmosphere.}

using \(P_{tot} = g m_{col} \rightarrow m_{col} = 1043.46 g/cm^2\)

\textbf{Final question: the energy flux of the sunshine reaching our
planet (``irradiance'') is}

\[ \mathscr{R} = \frac{4\pi R^2}{4\pi D^2} = \mathscr{F}^+_{\odot} \]

\textbf{with \(\mathscr{F}^+_{\odot} = \pi B(T^{\odot}_{eff})\) the
emergent solar flux, \(R\) the solar radius and \(D\) the sun-earth
distance so that the sunshine photon density at earth is}

\[ N_{phot} = \pi \frac{R^2}{D^2} N^{top}_{phot} \]

\textbf{with \(N^{top}_{phot}\) the photon density at the top of FALC
which you determined at the end of Section 1.1. Compare \(N_{phot}\) to
the particle density in the air around us, and to the local thermal
photon production derived from (2). Comments?}

    \begin{Verbatim}[commandchars=\\\{\}]
{\color{incolor}In [{\color{incolor}35}]:} \PY{n}{R} \PY{o}{=} \PY{n}{const}\PY{o}{.}\PY{n}{R\PYZus{}sun}\PY{o}{.}\PY{n}{cgs}\PY{o}{.}\PY{n}{value}
         \PY{n}{D} \PY{o}{=} \PY{n}{const}\PY{o}{.}\PY{n}{au}\PY{o}{.}\PY{n}{cgs}\PY{o}{.}\PY{n}{value}
         \PY{n}{Teff} \PY{o}{=} \PY{l+m+mi}{5770}
         \PY{n}{Nphot\PYZus{}sun} \PY{o}{=} \PY{p}{(}\PY{l+m+mi}{20}\PY{o}{*}\PY{n}{Teff}\PY{o}{*}\PY{o}{*}\PY{l+m+mi}{3}\PY{p}{)}\PY{o}{/}\PY{p}{(}\PY{l+m+mi}{2}\PY{o}{*}\PY{n}{np}\PY{o}{.}\PY{n}{pi}\PY{p}{)}
         \PY{n}{Nphot\PYZus{}earth} \PY{o}{=} \PY{n}{np}\PY{o}{.}\PY{n}{pi} \PY{o}{*} \PY{p}{(}\PY{n}{R}\PY{o}{*}\PY{o}{*}\PY{l+m+mi}{2}\PY{o}{/}\PY{n}{D}\PY{o}{*}\PY{o}{*}\PY{l+m+mi}{2}\PY{p}{)} \PY{o}{*} \PY{n}{Nphot\PYZus{}sun}
         
         \PY{n}{Tearth} \PY{o}{=} \PY{n}{T}\PY{p}{[}\PY{l+m+mi}{0}\PY{p}{]} 
         \PY{n}{Nphot\PYZus{}from\PYZus{}earth} \PY{o}{=} \PY{p}{(}\PY{l+m+mi}{20}\PY{o}{*}\PY{n}{Tearth}\PY{o}{*}\PY{o}{*}\PY{l+m+mi}{3}\PY{p}{)}\PY{o}{/}\PY{p}{(}\PY{l+m+mi}{2}\PY{o}{*}\PY{n}{np}\PY{o}{.}\PY{n}{pi}\PY{p}{)}
\end{Verbatim}

    \begin{Verbatim}[commandchars=\\\{\}]
{\color{incolor}In [{\color{incolor}36}]:} \PY{k}{print} \PY{l+s}{\PYZdq{}}\PY{l+s}{photon density at earth from the sun: \PYZob{}:.2\PYZcb{}}\PY{l+s}{\PYZdq{}}\PY{o}{.}\PY{n}{format}\PY{p}{(}\PY{n}{Nphot\PYZus{}earth}\PY{p}{)}
         \PY{k}{print} \PY{l+s}{\PYZdq{}}\PY{l+s}{particle density around us: \PYZob{}:.2\PYZcb{}}\PY{l+s}{\PYZdq{}}\PY{o}{.}\PY{n}{format}\PY{p}{(}\PY{p}{(}\PY{l+m+mi}{10}\PY{o}{*}\PY{o}{*}\PY{p}{(}\PY{n}{logN}\PY{p}{[}\PY{l+m+mi}{0}\PY{p}{]}\PY{p}{)}\PY{p}{)}\PY{p}{)}
         \PY{k}{print} \PY{l+s}{\PYZdq{}}\PY{l+s}{local thermal photon production: \PYZob{}:.2\PYZcb{}}\PY{l+s}{\PYZdq{}}\PY{o}{.}\PY{n}{format}\PY{p}{(}\PY{n}{Nphot\PYZus{}from\PYZus{}earth}\PY{p}{)}
\end{Verbatim}

    \begin{Verbatim}[commandchars=\\\{\}]
photon density at earth from the sun: 4.2e+07
particle density around us: 2.6e+19
local thermal photon production: 7.6e+07
    \end{Verbatim}

    The photon density is \textasciitilde{}11 orders of magnitude less than
the particle density at the surface of earth. In addition the local
thermal photon production is greater than that recieved by the sun.


    % Add a bibliography block to the postdoc
    
    
    
    \end{document}
